\batchmode %% Suppresses most terminal output.
\documentclass[../main.tex]{subfiles}
\setlength{\textwidth}{360pt}
\setlength{\textheight}{541pt}
\usepackage{latexsym}
\usepackage{ifthen}
% \usepackage{color}
%%%%%%%%%%%%%%%%%%%%%%%%%%%%%%%%%%%%%%%%%%%%%%%%%%%%%%%%%%%%%%%%%%%%%%%%%%%%%
% SWITCHES                                                                  %
%%%%%%%%%%%%%%%%%%%%%%%%%%%%%%%%%%%%%%%%%%%%%%%%%%%%%%%%%%%%%%%%%%%%%%%%%%%%%
\newboolean{shading} 
\setboolean{shading}{false}
\makeatletter
 %% this is needed only when inserted into the file, not when
 %% used as a package file.
%%%%%%%%%%%%%%%%%%%%%%%%%%%%%%%%%%%%%%%%%%%%%%%%%%%%%%%%%%%%%%%%%%%%%%%%%%%%%
%                                                                           %
% DEFINITIONS OF SYMBOL-PRODUCING COMMANDS                                  %
%                                                                           %
%    TLA+      LaTeX                                                        %
%    symbol    command                                                      %
%    ------    -------                                                      %
%    =>        \implies                                                     %
%    <:        \ltcolon                                                     %
%    :>        \colongt                                                     %
%    ==        \defeq                                                       %
%    ..        \dotdot                                                      %
%    ::        \coloncolon                                                  %
%    =|        \eqdash                                                      %
%    ++        \pp                                                          %
%    --        \mm                                                          %
%    **        \stst                                                        %
%    //        \slsl                                                        %
%    ^         \ct                                                          %
%    \A        \A                                                           %
%    \E        \E                                                           %
%    \AA       \AA                                                          %
%    \EE       \EE                                                          %
%%%%%%%%%%%%%%%%%%%%%%%%%%%%%%%%%%%%%%%%%%%%%%%%%%%%%%%%%%%%%%%%%%%%%%%%%%%%%
\newlength{\symlength}
\newcommand{\implies}{\Rightarrow}
\newcommand{\ltcolon}{\mathrel{<\!\!\mbox{:}}}
\newcommand{\colongt}{\mathrel{\!\mbox{:}\!\!>}}
\newcommand{\defeq}{\;\mathrel{\smash   %% keep this symbol from being too tall
    {{\stackrel{\scriptscriptstyle\Delta}{=}}}}\;}
\newcommand{\dotdot}{\mathrel{\ldotp\ldotp}}
\newcommand{\coloncolon}{\mathrel{::\;}}
\newcommand{\eqdash}{\mathrel = \joinrel \hspace{-.28em}|}
\newcommand{\pp}{\mathbin{++}}
\newcommand{\mm}{\mathbin{--}}
\newcommand{\stst}{*\!*}
\newcommand{\slsl}{/\!/}
\newcommand{\ct}{\hat{\hspace{.4em}}}
\newcommand{\A}{\forall}
\newcommand{\E}{\exists}
\renewcommand{\AA}{\makebox{$\raisebox{.05em}{\makebox[0pt][l]{%
   $\forall\hspace{-.517em}\forall\hspace{-.517em}\forall$}}%
   \forall\hspace{-.517em}\forall \hspace{-.517em}\forall\,$}}
\newcommand{\EE}{\makebox{$\raisebox{.05em}{\makebox[0pt][l]{%
   $\exists\hspace{-.517em}\exists\hspace{-.517em}\exists$}}%
   \exists\hspace{-.517em}\exists\hspace{-.517em}\exists\,$}}
\newcommand{\whileop}{\.{\stackrel
  {\mbox{\raisebox{-.3em}[0pt][0pt]{$\scriptscriptstyle+\;\,$}}}%
  {-\hspace{-.16em}\triangleright}}}

% Commands are defined to produce the upper-case keywords.
% Note that some have space after them.
\newcommand{\ASSUME}{\textsc{assume }}
\newcommand{\ASSUMPTION}{\textsc{assumption }}
\newcommand{\AXIOM}{\textsc{axiom }}
\newcommand{\BOOLEAN}{\textsc{boolean }}
\newcommand{\CASE}{\textsc{case }}
\newcommand{\CONSTANT}{\textsc{constant }}
\newcommand{\CONSTANTS}{\textsc{constants }}
\newcommand{\ELSE}{\settowidth{\symlength}{\THEN}%
   \makebox[\symlength][l]{\textsc{ else}}}
\newcommand{\EXCEPT}{\textsc{ except }}
\newcommand{\EXTENDS}{\textsc{extends }}
\newcommand{\FALSE}{\textsc{false}}
\newcommand{\IF}{\textsc{if }}
\newcommand{\IN}{\settowidth{\symlength}{\LET}%
   \makebox[\symlength][l]{\textsc{in}}}
\newcommand{\INSTANCE}{\textsc{instance }}
\newcommand{\LET}{\textsc{let }}
\newcommand{\LOCAL}{\textsc{local }}
\newcommand{\MODULE}{\textsc{module }}
\newcommand{\OTHER}{\textsc{other }}
\newcommand{\STRING}{\textsc{string}}
\newcommand{\THEN}{\textsc{ then }}
\newcommand{\THEOREM}{\textsc{theorem }}
\newcommand{\LEMMA}{\textsc{lemma }}
\newcommand{\PROPOSITION}{\textsc{proposition }}
\newcommand{\COROLLARY}{\textsc{corollary }}
\newcommand{\TRUE}{\textsc{true}}
\newcommand{\VARIABLE}{\textsc{variable }}
\newcommand{\VARIABLES}{\textsc{variables }}
\newcommand{\WITH}{\textsc{ with }}
\newcommand{\WF}{\textrm{WF}}
\newcommand{\SF}{\textrm{SF}}
\newcommand{\CHOOSE}{\textsc{choose }}
\newcommand{\ENABLED}{\textsc{enabled }}
\newcommand{\UNCHANGED}{\textsc{unchanged }}
\newcommand{\SUBSET}{\textsc{subset }}
\newcommand{\UNION}{\textsc{union }}
\newcommand{\DOMAIN}{\textsc{domain }}
% Added for tla2tex
\newcommand{\BY}{\textsc{by }}
\newcommand{\OBVIOUS}{\textsc{obvious }}
\newcommand{\HAVE}{\textsc{have }}
\newcommand{\QED}{\textsc{qed }}
\newcommand{\TAKE}{\textsc{take }}
\newcommand{\DEF}{\textsc{ def }}
\newcommand{\HIDE}{\textsc{hide }}
\newcommand{\RECURSIVE}{\textsc{recursive }}
\newcommand{\USE}{\textsc{use }}
\newcommand{\DEFINE}{\textsc{define }}
\newcommand{\PROOF}{\textsc{proof }}
\newcommand{\WITNESS}{\textsc{witness }}
\newcommand{\PICK}{\textsc{pick }}
\newcommand{\DEFS}{\textsc{defs }}
\newcommand{\PROVE}{\settowidth{\symlength}{\ASSUME}%
   \makebox[\symlength][l]{\textsc{prove}}\@s{-4.1}}%
  %% The \@s{-4.1) is a kludge added on 24 Oct 2009 [happy birthday, Ellen]
  %% so the correct alignment occurs if the user types
  %%   ASSUME X
  %%   PROVE  Y
  %% because it cancels the extra 4.1 pts added because of the 
  %% extra space after the PROVE.  This seems to works OK.
  %% However, the 4.1 equals Parameters.LaTeXLeftSpace(1) and
  %% should be changed if that method ever changes.
\newcommand{\SUFFICES}{\textsc{suffices }}
\newcommand{\NEW}{\textsc{new }}
\newcommand{\LAMBDA}{\textsc{lambda }}
\newcommand{\STATE}{\textsc{state }}
\newcommand{\ACTION}{\textsc{action }}
\newcommand{\TEMPORAL}{\textsc{temporal }}
\newcommand{\ONLY}{\textsc{only }}              %% added by LL on 2 Oct 2009
\newcommand{\OMITTED}{\textsc{omitted }}        %% added by LL on 31 Oct 2009
\newcommand{\@pfstepnum}[2]{\ensuremath{\langle#1\rangle}\textrm{#2}}
\newcommand{\bang}{\@s{1}\mbox{\small !}\@s{1}}
%% We should format || differently in PlusCal code than in TLA+ formulas.
\newcommand{\p@barbar}{\ifpcalsymbols
   \,\,\rule[-.25em]{.075em}{1em}\hspace*{.2em}\rule[-.25em]{.075em}{1em}\,\,%
   \else \,||\,\fi}
%% PlusCal keywords
\newcommand{\p@fair}{\textbf{fair }}
\newcommand{\p@semicolon}{\textbf{\,; }}
\newcommand{\p@algorithm}{\textbf{algorithm }}
\newcommand{\p@mmfair}{\textbf{-{}-fair }}
\newcommand{\p@mmalgorithm}{\textbf{-{}-algorithm }}
\newcommand{\p@assert}{\textbf{assert }}
\newcommand{\p@await}{\textbf{await }}
\newcommand{\p@begin}{\textbf{begin }}
\newcommand{\p@end}{\textbf{end }}
\newcommand{\p@call}{\textbf{call }}
\newcommand{\p@define}{\textbf{define }}
\newcommand{\p@do}{\textbf{ do }}
\newcommand{\p@either}{\textbf{either }}
\newcommand{\p@or}{\textbf{or }}
\newcommand{\p@goto}{\textbf{goto }}
\newcommand{\p@if}{\textbf{if }}
\newcommand{\p@then}{\,\,\textbf{then }}
\newcommand{\p@else}{\ifcsyntax \textbf{else } \else \,\,\textbf{else }\fi}
\newcommand{\p@elsif}{\,\,\textbf{elsif }}
\newcommand{\p@macro}{\textbf{macro }}
\newcommand{\p@print}{\textbf{print }}
\newcommand{\p@procedure}{\textbf{procedure }}
\newcommand{\p@process}{\textbf{process }}
\newcommand{\p@return}{\textbf{return}}
\newcommand{\p@skip}{\textbf{skip}}
\newcommand{\p@variable}{\textbf{variable }}
\newcommand{\p@variables}{\textbf{variables }}
\newcommand{\p@while}{\textbf{while }}
\newcommand{\p@when}{\textbf{when }}
\newcommand{\p@with}{\textbf{with }}
\newcommand{\p@lparen}{\textbf{(\,\,}}
\newcommand{\p@rparen}{\textbf{\,\,) }}   
\newcommand{\p@lbrace}{\textbf{\{\,\,}}   
\newcommand{\p@rbrace}{\textbf{\,\,\} }}

%%%%%%%%%%%%%%%%%%%%%%%%%%%%%%%%%%%%%%%%%%%%%%%%%%%%%%%%%
% REDEFINE STANDARD COMMANDS TO MAKE THEM FORMAT BETTER %
%                                                       %
% We redefine \in and \notin                            %
%%%%%%%%%%%%%%%%%%%%%%%%%%%%%%%%%%%%%%%%%%%%%%%%%%%%%%%%%
\renewcommand{\_}{\rule{.4em}{.06em}\hspace{.05em}}
\newlength{\equalswidth}
\let\oldin=\in
\let\oldnotin=\notin
\renewcommand{\in}{%
   {\settowidth{\equalswidth}{$\.{=}$}\makebox[\equalswidth][c]{$\oldin$}}}
\renewcommand{\notin}{%
   {\settowidth{\equalswidth}{$\.{=}$}\makebox[\equalswidth]{$\oldnotin$}}}


%%%%%%%%%%%%%%%%%%%%%%%%%%%%%%%%%%%%%%%%%%%%%%%%%%%%
%                                                  %
% HORIZONTAL BARS:                                 %
%                                                  %
%   \moduleLeftDash    |~~~~~~~~~~                 %
%   \moduleRightDash    ~~~~~~~~~~|                %
%   \midbar            |----------|                %
%   \bottombar         |__________|                %
%%%%%%%%%%%%%%%%%%%%%%%%%%%%%%%%%%%%%%%%%%%%%%%%%%%%
\newlength{\charwidth}\settowidth{\charwidth}{{\small\tt M}}
\newlength{\boxrulewd}\setlength{\boxrulewd}{.4pt}
\newlength{\boxlineht}\setlength{\boxlineht}{.5\baselineskip}
\newcommand{\boxsep}{\charwidth}
\newlength{\boxruleht}\setlength{\boxruleht}{.5ex}
\newlength{\boxruledp}\setlength{\boxruledp}{-\boxruleht}
\addtolength{\boxruledp}{\boxrulewd}
\newcommand{\boxrule}{\leaders\hrule height \boxruleht depth \boxruledp
                      \hfill\mbox{}}
\newcommand{\@computerule}{%
  \setlength{\boxruleht}{.5ex}%
  \setlength{\boxruledp}{-\boxruleht}%
  \addtolength{\boxruledp}{\boxrulewd}}

\newcommand{\bottombar}{\hspace{-\boxsep}%
  \raisebox{-\boxrulewd}[0pt][0pt]{\rule[.5ex]{\boxrulewd}{\boxlineht}}%
  \boxrule
  \raisebox{-\boxrulewd}[0pt][0pt]{%
      \rule[.5ex]{\boxrulewd}{\boxlineht}}\hspace{-\boxsep}\vspace{0pt}}

\newcommand{\moduleLeftDash}%
   {\hspace*{-\boxsep}%
     \raisebox{-\boxlineht}[0pt][0pt]{\rule[.5ex]{\boxrulewd
               }{\boxlineht}}%
    \boxrule\hspace*{.4em }}

\newcommand{\moduleRightDash}%
    {\hspace*{.4em}\boxrule
    \raisebox{-\boxlineht}[0pt][0pt]{\rule[.5ex]{\boxrulewd
               }{\boxlineht}}\hspace{-\boxsep}}%\vspace{.2em}

\newcommand{\midbar}{\hspace{-\boxsep}\raisebox{-.5\boxlineht}[0pt][0pt]{%
   \rule[.5ex]{\boxrulewd}{\boxlineht}}\boxrule\raisebox{-.5\boxlineht%
   }[0pt][0pt]{\rule[.5ex]{\boxrulewd}{\boxlineht}}\hspace{-\boxsep}}

%%%%%%%%%%%%%%%%%%%%%%%%%%%%%%%%%%%%%%%%%%%%%%%%%%%%%%%%%%%%%%%%%%%%%%%%%%%%%
% FORMATING COMMANDS                                                        %
%%%%%%%%%%%%%%%%%%%%%%%%%%%%%%%%%%%%%%%%%%%%%%%%%%%%%%%%%%%%%%%%%%%%%%%%%%%%%

%%%%%%%%%%%%%%%%%%%%%%%%%%%%%%%%%%%%%%%%%%%%%%%%%%%%%%%%%%%%%%%%%%%%%%%%%%%%%
% PLUSCAL SHADING                                                           %
%%%%%%%%%%%%%%%%%%%%%%%%%%%%%%%%%%%%%%%%%%%%%%%%%%%%%%%%%%%%%%%%%%%%%%%%%%%%%

% The TeX pcalshading switch is set on to cause PlusCal shading to be
% performed.  This changes the behavior of the following commands and
% environments to cause full-width shading to be performed on all lines.
% 
%   \tstrut \@x cpar mcom \@pvspace
% 
% The TeX pcalsymbols switch is turned on when typesetting a PlusCal algorithm,
% whether or not shading is being performed.  It causes symbols (other than
% parentheses and braces and PlusCal-only keywords) that should be typeset
% differently depending on whether they are in an algorithm to be typeset
% appropriately.  Currently, the only such symbol is "||".
%
% The TeX csyntax switch is turned on when typesetting a PlusCal algorithm in
% c-syntax.  This allows symbols to be format differently in the two syntaxes.
% The "else" keyword is the only one that is.

\newif\ifpcalshading \pcalshadingfalse
\newif\ifpcalsymbols \pcalsymbolsfalse
\newif\ifcsyntax     \csyntaxtrue

% The \@pvspace command makes a vertical space.  It uses \vspace
% except with \ifpcalshading, in which case it sets \pvcalvspace
% and the space is added by a following \@x command.
%
\newlength{\pcalvspace}\setlength{\pcalvspace}{0pt}%
\newcommand{\@pvspace}[1]{%
  \ifpcalshading
     \par\global\setlength{\pcalvspace}{#1}%
  \else
     \par\vspace{#1}%
  \fi
}

% The lcom environment was changed to set \lcomindent equal to
% the indentation it produces.  This length is used by the
% cpar environment to make shading extend for the full width
% of the line.  This assumes that lcom environments are not
% nested.  I hope TLATeX does not nest them.
%
\newlength{\lcomindent}%
\setlength{\lcomindent}{0pt}%

%\tstrut: A strut to produce inter-paragraph space in a comment.
%\rstrut: A strut to extend the bottom of a one-line comment so
%         there's no break in the shading between comments on 
%         successive lines.
\newcommand\tstrut%
  {\raisebox{\vshadelen}{\raisebox{-.25em}{\rule{0pt}{1.15em}}}%
   \global\setlength{\vshadelen}{0pt}}
\newcommand\rstrut{\raisebox{-.25em}{\rule{0pt}{1.15em}}%
 \global\setlength{\vshadelen}{0pt}}


% \.{op} formats operator op in math mode with empty boxes on either side.
% Used because TeX otherwise vary the amount of space it leaves around op.
\renewcommand{\.}[1]{\ensuremath{\mbox{}#1\mbox{}}}

% \@s{n} produces an n-point space
\newcommand{\@s}[1]{\hspace{#1pt}}           

% \@x{txt} starts a specification line in the beginning with txt
% in the final LaTeX source.
\newlength{\@xlen}
\newcommand\xtstrut%
  {\setlength{\@xlen}{1.05em}%
   \addtolength{\@xlen}{\pcalvspace}%
    \raisebox{\vshadelen}{\raisebox{-.25em}{\rule{0pt}{\@xlen}}}%
   \global\setlength{\vshadelen}{0pt}%
   \global\setlength{\pcalvspace}{0pt}}

\newcommand{\@x}[1]{\par
  \ifpcalshading
  \makebox[0pt][l]{\shadebox{\xtstrut\hspace*{\textwidth}}}%
  \fi
  \mbox{$\mbox{}#1\mbox{}$}}  

% \@xx{txt} continues a specification line with the text txt.
\newcommand{\@xx}[1]{\mbox{$\mbox{}#1\mbox{}$}}  

% \@y{cmt} produces a one-line comment.
\newcommand{\@y}[1]{\mbox{\footnotesize\hspace{.65em}%
  \ifthenelse{\boolean{shading}}{%
      \shadebox{#1\hspace{-\the\lastskip}\rstrut}}%
               {#1\hspace{-\the\lastskip}\rstrut}}}

% \@z{cmt} produces a zero-width one-line comment.
\newcommand{\@z}[1]{\makebox[0pt][l]{\footnotesize
  \ifthenelse{\boolean{shading}}{%
      \shadebox{#1\hspace{-\the\lastskip}\rstrut}}%
               {#1\hspace{-\the\lastskip}\rstrut}}}


% \@w{str} produces the TLA+ string "str".
\newcommand{\@w}[1]{\textsf{``{#1}''}}             


%%%%%%%%%%%%%%%%%%%%%%%%%%%%%%%%%%%%%%%%%%%%%%%%%%%%%%%%%%%%%%%%%%%%%%%%%%%%%
% SHADING                                                                   %
%%%%%%%%%%%%%%%%%%%%%%%%%%%%%%%%%%%%%%%%%%%%%%%%%%%%%%%%%%%%%%%%%%%%%%%%%%%%%
\def\graymargin{1}
  % The number of points of margin in the shaded box.

% \definecolor{boxshade}{gray}{.85}
% Defines the darkness of the shading: 1 = white, 0 = black
% Added by TLATeX only if needed.

% \shadebox{txt} puts txt in a shaded box.
\newlength{\templena}
\newlength{\templenb}
\newsavebox{\tempboxa}
\newcommand{\shadebox}[1]{{\setlength{\fboxsep}{\graymargin pt}%
     \savebox{\tempboxa}{#1}%
     \settoheight{\templena}{\usebox{\tempboxa}}%
     \settodepth{\templenb}{\usebox{\tempboxa}}%
     \hspace*{-\fboxsep}\raisebox{0pt}[\templena][\templenb]%
        {\colorbox{boxshade}{\usebox{\tempboxa}}}\hspace*{-\fboxsep}}}

% \vshade{n} makes an n-point inter-paragraph space, with
%  shading if the `shading' flag is true.
\newlength{\vshadelen}
\setlength{\vshadelen}{0pt}
\newcommand{\vshade}[1]{\ifthenelse{\boolean{shading}}%
   {\global\setlength{\vshadelen}{#1pt}}%
   {\vspace{#1pt}}}

\newlength{\boxwidth}
\newlength{\multicommentdepth}

%%%%%%%%%%%%%%%%%%%%%%%%%%%%%%%%%%%%%%%%%%%%%%%%%%%%%%%%%%%%%%%%%%%%%%%%%%%%%
% THE cpar ENVIRONMENT                                                      %
% ^^^^^^^^^^^^^^^^^^^^                                                      %
% The LaTeX input                                                           %
%                                                                           %
%   \begin{cpar}{pop}{nest}{isLabel}{d}{e}{arg6}                            %
%     XXXXXXXXXXXXXXX                                                       %
%     XXXXXXXXXXXXXXX                                                       %
%     XXXXXXXXXXXXXXX                                                       %
%   \end{cpar}                                                              %
%                                                                           %
% produces one of two possible results.  If isLabel is the letter "T",      %
% it produces the following, where [label] is the result of typesetting     %
% arg6 in an LR box, and d is is a number representing a distance in        %
% points.                                                                   %
%                                                                           %
%   prevailing |<-- d -->[label]<- e ->XXXXXXXXXXXXXXX                      %
%         left |                       XXXXXXXXXXXXXXX                      %
%       margin |                       XXXXXXXXXXXXXXX                      %
%                                                                           %
% If isLabel is the letter "F", then it produces                            %
%                                                                           %
%   prevailing |<-- d -->XXXXXXXXXXXXXXXXXXXXXXX                            %
%         left |         <- e ->XXXXXXXXXXXXXXXX                            %
%       margin |                XXXXXXXXXXXXXXXX                            %
%                                                                           %
% where d and e are numbers representing distances in points.               %
%                                                                           %
% The prevailing left margin is the one in effect before the most recent    %
% pop (argument 1) cpar environments with "T" as the nest argument, where   %
% pop is a number \geq 0.                                                   %
%                                                                           %
% If the nest argument is the letter "T", then the prevailing left          %
% margin is moved to the left of the second (and following) lines of        %
% X's.  Otherwise, the prevailing left margin is left unchanged.            %
%                                                                           %
% An \unnest{n} command moves the prevailing left margin to where it was    %
% before the most recent n cpar environments with "T" as the nesting        %
% argument.                                                                 %
%                                                                           %
% The environment leaves no vertical space above or below it, or between    %
% its paragraphs.  (TLATeX inserts the proper amount of vertical space.)    %
%%%%%%%%%%%%%%%%%%%%%%%%%%%%%%%%%%%%%%%%%%%%%%%%%%%%%%%%%%%%%%%%%%%%%%%%%%%%%

\newcounter{pardepth}
\setcounter{pardepth}{0}

% \setgmargin{txt} defines \gmarginN to be txt, where N is \roman{pardepth}.
% \thegmargin equals \gmarginN, where N is \roman{pardepth}.
\newcommand{\setgmargin}[1]{%
  \expandafter\xdef\csname gmargin\roman{pardepth}\endcsname{#1}}
\newcommand{\thegmargin}{\csname gmargin\roman{pardepth}\endcsname}
\newcommand{\gmargin}{0pt}

\newsavebox{\tempsbox}

\newlength{\@cparht}
\newlength{\@cpardp}
\newenvironment{cpar}[6]{%
  \addtocounter{pardepth}{-#1}%
  \ifthenelse{\boolean{shading}}{\par\begin{lrbox}{\tempsbox}%
                                 \begin{minipage}[t]{\linewidth}}{}%
  \begin{list}{}{%
     \edef\temp{\thegmargin}
     \ifthenelse{\equal{#3}{T}}%
       {\settowidth{\leftmargin}{\hspace{\temp}\footnotesize #6\hspace{#5pt}}%
        \addtolength{\leftmargin}{#4pt}}%
       {\setlength{\leftmargin}{#4pt}%
        \addtolength{\leftmargin}{#5pt}%
        \addtolength{\leftmargin}{\temp}%
        \setlength{\itemindent}{-#5pt}}%
      \ifthenelse{\equal{#2}{T}}{\addtocounter{pardepth}{1}%
                                 \setgmargin{\the\leftmargin}}{}%
      \setlength{\labelwidth}{0pt}%
      \setlength{\labelsep}{0pt}%
      \setlength{\itemindent}{-\leftmargin}%
      \setlength{\topsep}{0pt}%
      \setlength{\parsep}{0pt}%
      \setlength{\partopsep}{0pt}%
      \setlength{\parskip}{0pt}%
      \setlength{\itemsep}{0pt}
      \setlength{\itemindent}{#4pt}%
      \addtolength{\itemindent}{-\leftmargin}}%
   \ifthenelse{\equal{#3}{T}}%
      {\item[\tstrut\footnotesize \hspace{\temp}{#6}\hspace{#5pt}]
        }%
      {\item[\tstrut\hspace{\temp}]%
         }%
   \footnotesize}
 {\hspace{-\the\lastskip}\tstrut
 \end{list}%
  \ifthenelse{\boolean{shading}}%
          {\end{minipage}%
           \end{lrbox}%
           \ifpcalshading
             \setlength{\@cparht}{\ht\tempsbox}%
             \setlength{\@cpardp}{\dp\tempsbox}%
             \addtolength{\@cparht}{.15em}%
             \addtolength{\@cpardp}{.2em}%
             \addtolength{\@cparht}{\@cpardp}%
            % I don't know what's going on here.  I want to add a
            % \pcalvspace high shaded line, but I don't know how to
            % do it.  A little trial and error shows that the following
            % does a reasonable job approximating that, eliminating
            % the line if \pcalvspace is small.
            \addtolength{\@cparht}{\pcalvspace}%
             \ifdim \pcalvspace > .8em
               \addtolength{\pcalvspace}{-.2em}%
               \hspace*{-\lcomindent}%
               \shadebox{\rule{0pt}{\pcalvspace}\hspace*{\textwidth}}\par
               \global\setlength{\pcalvspace}{0pt}%
               \fi
             \hspace*{-\lcomindent}%
             \makebox[0pt][l]{\raisebox{-\@cpardp}[0pt][0pt]{%
                 \shadebox{\rule{0pt}{\@cparht}\hspace*{\textwidth}}}}%
             \hspace*{\lcomindent}\usebox{\tempsbox}%
             \par
           \else
             \shadebox{\usebox{\tempsbox}}\par
           \fi}%
           {}%
  }

%%%%%%%%%%%%%%%%%%%%%%%%%%%%%%%%%%%%%%%%%%%%%%%%%%%%%%%%%%%%%%%%%%%%%%%%%%%%%%
% THE ppar ENVIRONMENT                                                       %
% ^^^^^^^^^^^^^^^^^^^^                                                       %
% The environment                                                            %
%                                                                            %
%   \begin{ppar} ... \end{ppar}                                              %
%                                                                            %
% is equivalent to                                                           %
%                                                                            %
%   \begin{cpar}{0}{F}{F}{0}{0}{} ... \end{cpar}                             %
%                                                                            %
% The environment is put around each line of the output for a PlusCal        %
% algorithm.                                                                 %
%%%%%%%%%%%%%%%%%%%%%%%%%%%%%%%%%%%%%%%%%%%%%%%%%%%%%%%%%%%%%%%%%%%%%%%%%%%%%%
%\newenvironment{ppar}{%
%  \ifthenelse{\boolean{shading}}{\par\begin{lrbox}{\tempsbox}%
%                                 \begin{minipage}[t]{\linewidth}}{}%
%  \begin{list}{}{%
%     \edef\temp{\thegmargin}
%        \setlength{\leftmargin}{0pt}%
%        \addtolength{\leftmargin}{\temp}%
%        \setlength{\itemindent}{0pt}%
%      \setlength{\labelwidth}{0pt}%
%      \setlength{\labelsep}{0pt}%
%      \setlength{\itemindent}{-\leftmargin}%
%      \setlength{\topsep}{0pt}%
%      \setlength{\parsep}{0pt}%
%      \setlength{\partopsep}{0pt}%
%      \setlength{\parskip}{0pt}%
%      \setlength{\itemsep}{0pt}
%      \setlength{\itemindent}{0pt}%
%      \addtolength{\itemindent}{-\leftmargin}}%
%      \item[\tstrut\hspace{\temp}]}%
% {\hspace{-\the\lastskip}\tstrut
% \end{list}%
%  \ifthenelse{\boolean{shading}}{\end{minipage}  
%                                 \end{lrbox}%
%                                 \shadebox{\usebox{\tempsbox}}\par}{}%
%  }

 %%% TESTING
 \newcommand{\xtest}[1]{\par
 \makebox[0pt][l]{\shadebox{\xtstrut\hspace*{\textwidth}}}%
 \mbox{$\mbox{}#1\mbox{}$}} 

% \newcommand{\xxtest}[1]{\par
% \makebox[0pt][l]{\shadebox{\xtstrut{#1}\hspace*{\textwidth}}}%
% \mbox{$\mbox{}#1\mbox{}$}} 

%\newlength{\pcalvspace}
%\setlength{\pcalvspace}{0pt}
% \newlength{\xxtestlen}
% \setlength{\xxtestlen}{0pt}
% \newcommand\xtstrut%
%   {\setlength{\xxtestlen}{1.15em}%
%    \addtolength{\xxtestlen}{\pcalvspace}%
%     \raisebox{\vshadelen}{\raisebox{-.25em}{\rule{0pt}{\xxtestlen}}}%
%    \global\setlength{\vshadelen}{0pt}%
%    \global\setlength{\pcalvspace}{0pt}}
   
   %%%% TESTING
   
   %% The xcpar environment
   %%  Note: overloaded use of \pcalvspace for testing.
   %%
%   \newlength{\xcparht}%
%   \newlength{\xcpardp}%
   
%   \newenvironment{xcpar}[6]{%
%  \addtocounter{pardepth}{-#1}%
%  \ifthenelse{\boolean{shading}}{\par\begin{lrbox}{\tempsbox}%
%                                 \begin{minipage}[t]{\linewidth}}{}%
%  \begin{list}{}{%
%     \edef\temp{\thegmargin}%
%     \ifthenelse{\equal{#3}{T}}%
%       {\settowidth{\leftmargin}{\hspace{\temp}\footnotesize #6\hspace{#5pt}}%
%        \addtolength{\leftmargin}{#4pt}}%
%       {\setlength{\leftmargin}{#4pt}%
%        \addtolength{\leftmargin}{#5pt}%
%        \addtolength{\leftmargin}{\temp}%
%        \setlength{\itemindent}{-#5pt}}%
%      \ifthenelse{\equal{#2}{T}}{\addtocounter{pardepth}{1}%
%                                 \setgmargin{\the\leftmargin}}{}%
%      \setlength{\labelwidth}{0pt}%
%      \setlength{\labelsep}{0pt}%
%      \setlength{\itemindent}{-\leftmargin}%
%      \setlength{\topsep}{0pt}%
%      \setlength{\parsep}{0pt}%
%      \setlength{\partopsep}{0pt}%
%      \setlength{\parskip}{0pt}%
%      \setlength{\itemsep}{0pt}%
%      \setlength{\itemindent}{#4pt}%
%      \addtolength{\itemindent}{-\leftmargin}}%
%   \ifthenelse{\equal{#3}{T}}%
%      {\item[\xtstrut\footnotesize \hspace{\temp}{#6}\hspace{#5pt}]%
%        }%
%      {\item[\xtstrut\hspace{\temp}]%
%         }%
%   \footnotesize}
% {\hspace{-\the\lastskip}\tstrut
% \end{list}%
%  \ifthenelse{\boolean{shading}}{\end{minipage}  
%                                 \end{lrbox}%
%                                 \setlength{\xcparht}{\ht\tempsbox}%
%                                 \setlength{\xcpardp}{\dp\tempsbox}%
%                                 \addtolength{\xcparht}{.15em}%
%                                 \addtolength{\xcpardp}{.2em}%
%                                 \addtolength{\xcparht}{\xcpardp}%
%                                 \hspace*{-\lcomindent}%
%                                 \makebox[0pt][l]{\raisebox{-\xcpardp}[0pt][0pt]{%
%                                      \shadebox{\rule{0pt}{\xcparht}\hspace*{\textwidth}}}}%
%                                 \hspace*{\lcomindent}\usebox{\tempsbox}%
%                                 \par}{}%
%  }
%  
% \newlength{\xmcomlen}
%\newenvironment{xmcom}[1]{%
%  \setcounter{pardepth}{0}%
%  \hspace{.65em}%
%  \begin{lrbox}{\alignbox}\sloppypar%
%      \setboolean{shading}{false}%
%      \setlength{\boxwidth}{#1pt}%
%      \addtolength{\boxwidth}{-.65em}%
%      \begin{minipage}[t]{\boxwidth}\footnotesize
%      \parskip=0pt\relax}%
%       {\end{minipage}\end{lrbox}%
%       \setlength{\xmcomlen}{\textwidth}%
%       \addtolength{\xmcomlen}{-\wd\alignbox}%
%       \settodepth{\alignwidth}{\usebox{\alignbox}}%
%       \global\setlength{\multicommentdepth}{\alignwidth}%
%       \setlength{\boxwidth}{\alignwidth}%
%       \global\addtolength{\alignwidth}{-\maxdepth}%
%       \addtolength{\boxwidth}{.1em}%
%       \raisebox{0pt}[0pt][0pt]{%
%        \ifthenelse{\boolean{shading}}%
%          {\hspace*{-\xmcomlen}\shadebox{\rule[-\boxwidth]{0pt}{0pt}%
%                                 \hspace*{\xmcomlen}\usebox{\alignbox}}}%
%          {\usebox{\alignbox}}}%
%       \vspace*{\alignwidth}\pagebreak[0]\vspace{-\alignwidth}\par}
% % a multi-line comment, whose first argument is its width in points.
%  
   
%%%%%%%%%%%%%%%%%%%%%%%%%%%%%%%%%%%%%%%%%%%%%%%%%%%%%%%%%%%%%%%%%%%%%%%%%%%%%%
% THE lcom ENVIRONMENT                                                       %
% ^^^^^^^^^^^^^^^^^^^^                                                       %
% A multi-line comment with no text to its left is typeset in an lcom        % 
% environment, whose argument is a number representing the indentation       % 
% of the left margin, in points.  All the text of the comment should be      % 
% inside cpar environments.                                                  % 
%%%%%%%%%%%%%%%%%%%%%%%%%%%%%%%%%%%%%%%%%%%%%%%%%%%%%%%%%%%%%%%%%%%%%%%%%%%%%%
\newenvironment{lcom}[1]{%
  \setlength{\lcomindent}{#1pt} % Added for PlusCal handling.
  \par\vspace{.2em}%
  \sloppypar
  \setcounter{pardepth}{0}%
  \footnotesize
  \begin{list}{}{%
    \setlength{\leftmargin}{#1pt}
    \setlength{\labelwidth}{0pt}%
    \setlength{\labelsep}{0pt}%
    \setlength{\itemindent}{0pt}%
    \setlength{\topsep}{0pt}%
    \setlength{\parsep}{0pt}%
    \setlength{\partopsep}{0pt}%
    \setlength{\parskip}{0pt}}
    \item[]}%
  {\end{list}\vspace{.3em}\setlength{\lcomindent}{0pt}%
 }


%%%%%%%%%%%%%%%%%%%%%%%%%%%%%%%%%%%%%%%%%%%%%%%%%%%%%%%%%%%%%%%%%%%%%%%%%%%%%
% THE mcom ENVIRONMENT AND \mutivspace COMMAND                              %
% ^^^^^^^^^^^^^^^^^^^^^^^^^^^^^^^^^^^^^^^^^^^^                              %
%                                                                           %
% A part of the spec containing a right-comment of the form                 %
%                                                                           %
%      xxxx (*************)                                                 %
%      yyyy (* ccccccccc *)                                                 %
%      ...  (* ccccccccc *)                                                 %
%           (* ccccccccc *)                                                 %
%           (* ccccccccc *)                                                 %
%           (*************)                                                 %
%                                                                           %
% is typeset by                                                             %
%                                                                           %
%     XXXX \begin{mcom}{d}                                                  %
%            CCCC ... CCC                                                   %
%          \end{mcom}                                                       %
%     YYYY ...                                                              %
%     \multivspace{n}                                                       %
%                                                                           %
% where the number d is the width in points of the comment, n is the        %
% number of xxxx, yyyy, ...  lines to the left of the comment.              %
% All the text of the comment should be typeset in cpar environments.       %
%                                                                           %
% This puts the comment into a single box (so no page breaks can occur      %
% within it).  The entire box is shaded iff the shading flag is true.       %
%%%%%%%%%%%%%%%%%%%%%%%%%%%%%%%%%%%%%%%%%%%%%%%%%%%%%%%%%%%%%%%%%%%%%%%%%%%%%
\newlength{\xmcomlen}%
\newenvironment{mcom}[1]{%
  \setcounter{pardepth}{0}%
  \hspace{.65em}%
  \begin{lrbox}{\alignbox}\sloppypar%
      \setboolean{shading}{false}%
      \setlength{\boxwidth}{#1pt}%
      \addtolength{\boxwidth}{-.65em}%
      \begin{minipage}[t]{\boxwidth}\footnotesize
      \parskip=0pt\relax}%
       {\end{minipage}\end{lrbox}%
       \setlength{\xmcomlen}{\textwidth}%       % For PlusCal shading
       \addtolength{\xmcomlen}{-\wd\alignbox}%  % For PlusCal shading
       \settodepth{\alignwidth}{\usebox{\alignbox}}%
       \global\setlength{\multicommentdepth}{\alignwidth}%
       \setlength{\boxwidth}{\alignwidth}%      % For PlusCal shading
       \global\addtolength{\alignwidth}{-\maxdepth}%
       \addtolength{\boxwidth}{.1em}%           % For PlusCal shading
      \raisebox{0pt}[0pt][0pt]{%
        \ifthenelse{\boolean{shading}}%
          {\ifpcalshading
             \hspace*{-\xmcomlen}%
             \shadebox{\rule[-\boxwidth]{0pt}{0pt}\hspace*{\xmcomlen}%
                          \usebox{\alignbox}}%
           \else
             \shadebox{\usebox{\alignbox}}
           \fi
          }%
          {\usebox{\alignbox}}}%
       \vspace*{\alignwidth}\pagebreak[0]\vspace{-\alignwidth}\par}
 % a multi-line comment, whose first argument is its width in points.


% \multispace{n} produces the vertical space indicated by "|"s in 
% this situation
%   
%     xxxx (*************)
%     xxxx (* ccccccccc *)
%      |   (* ccccccccc *)
%      |   (* ccccccccc *)
%      |   (* ccccccccc *)
%      |   (*************)
%
% where n is the number of "xxxx" lines.
\newcommand{\multivspace}[1]{\addtolength{\multicommentdepth}{-#1\baselineskip}%
 \addtolength{\multicommentdepth}{1.2em}%
 \ifthenelse{\lengthtest{\multicommentdepth > 0pt}}%
    {\par\vspace{\multicommentdepth}\par}{}}

%\newenvironment{hpar}[2]{%
%  \begin{list}{}{\setlength{\leftmargin}{#1pt}%
%                 \addtolength{\leftmargin}{#2pt}%
%                 \setlength{\itemindent}{-#2pt}%
%                 \setlength{\topsep}{0pt}%
%                 \setlength{\parsep}{0pt}%
%                 \setlength{\partopsep}{0pt}%
%                 \setlength{\parskip}{0pt}%
%                 \addtolength{\labelsep}{0pt}}%
%  \item[]\footnotesize}{\end{list}}
%    %%%%%%%%%%%%%%%%%%%%%%%%%%%%%%%%%%%%%%%%%%%%%%%%%%%%%%%%%%%%%%%%%%%%%%%%
%    % Typesets a sequence of paragraphs like this:                         %
%    %                                                                      %
%    %      left |<-- d1 --> XXXXXXXXXXXXXXXXXXXXXXXX                       %
%    %    margin |           <- d2 -> XXXXXXXXXXXXXXX                       %
%    %           |                    XXXXXXXXXXXXXXX                       %
%    %           |                                                          %
%    %           |                    XXXXXXXXXXXXXXX                       %
%    %           |                    XXXXXXXXXXXXXXX                       %
%    %                                                                      %
%    % where d1 = #1pt and d2 = #2pt, but with no vspace between            %
%    % paragraphs.                                                          %
%    %%%%%%%%%%%%%%%%%%%%%%%%%%%%%%%%%%%%%%%%%%%%%%%%%%%%%%%%%%%%%%%%%%%%%%%%

%%%%%%%%%%%%%%%%%%%%%%%%%%%%%%%%%%%%%%%%%%%%%%%%%%%%%%%%%%%%%%%%%%%%%%
% Commands for repeated characters that produce dashes.              %
%%%%%%%%%%%%%%%%%%%%%%%%%%%%%%%%%%%%%%%%%%%%%%%%%%%%%%%%%%%%%%%%%%%%%%
% \raisedDash{wd}{ht}{thk} makes a horizontal line wd characters wide, 
% raised a distance ht ex's above the baseline, with a thickness of 
% thk em's.
\newcommand{\raisedDash}[3]{\raisebox{#2ex}{\setlength{\alignwidth}{.5em}%
  \rule{#1\alignwidth}{#3em}}}

% The following commands take a single argument n and produce the
% output for n repeated characters, as follows
%   \cdash:    -
%   \tdash:    ~
%   \ceqdash:  =
%   \usdash:   _
\newcommand{\cdash}[1]{\raisedDash{#1}{.5}{.04}}
\newcommand{\usdash}[1]{\raisedDash{#1}{0}{.04}}
\newcommand{\ceqdash}[1]{\raisedDash{#1}{.5}{.08}}
\newcommand{\tdash}[1]{\raisedDash{#1}{1}{.08}}

\newlength{\spacewidth}
\setlength{\spacewidth}{.2em}
\newcommand{\e}[1]{\hspace{#1\spacewidth}}
%% \e{i} produces space corresponding to i input spaces.


%% Alignment-file Commands

\newlength{\alignboxwidth}
\newlength{\alignwidth}
\newsavebox{\alignbox}

% \al{i}{j}{txt} is used in the alignment file to put "%{i}{j}{wd}"
% in the log file, where wd is the width of the line up to that point,
% and txt is the following text.
\newcommand{\al}[3]{%
  \typeout{\%{#1}{#2}{\the\alignwidth}}%
  \cl{#3}}

%% \cl{txt} continues a specification line in the alignment file
%% with text txt.
\newcommand{\cl}[1]{%
  \savebox{\alignbox}{\mbox{$\mbox{}#1\mbox{}$}}%
  \settowidth{\alignboxwidth}{\usebox{\alignbox}}%
  \addtolength{\alignwidth}{\alignboxwidth}%
  \usebox{\alignbox}}

% \fl{txt} in the alignment file begins a specification line that
% starts with the text txt.
\newcommand{\fl}[1]{%
  \par
  \savebox{\alignbox}{\mbox{$\mbox{}#1\mbox{}$}}%
  \settowidth{\alignwidth}{\usebox{\alignbox}}%
  \usebox{\alignbox}}



  
%%%%%%%%%%%%%%%%%%%%%%%%%%%%%%%%%%%%%%%%%%%%%%%%%%%%%%%%%%%%%%%%%%%%%%%%%%%%%
% Ordinarily, TeX typesets letters in math mode in a special math italic    %
% font.  This makes it typeset "it" to look like the product of the         %
% variables i and t, rather than like the word "it".  The following         %
% commands tell TeX to use an ordinary italic font instead.                 %
%%%%%%%%%%%%%%%%%%%%%%%%%%%%%%%%%%%%%%%%%%%%%%%%%%%%%%%%%%%%%%%%%%%%%%%%%%%%%
\ifx\documentclass\undefined
\else
  \DeclareSymbolFont{tlaitalics}{\encodingdefault}{cmr}{m}{it}
  \let\itfam\symtlaitalics
\fi

\makeatletter
\newcommand{\tlx@c}{\c@tlx@ctr\advance\c@tlx@ctr\@ne}
\newcounter{tlx@ctr}
\c@tlx@ctr=\itfam \multiply\c@tlx@ctr"100\relax \advance\c@tlx@ctr "7061\relax
\mathcode`a=\tlx@c \mathcode`b=\tlx@c \mathcode`c=\tlx@c \mathcode`d=\tlx@c
\mathcode`e=\tlx@c \mathcode`f=\tlx@c \mathcode`g=\tlx@c \mathcode`h=\tlx@c
\mathcode`i=\tlx@c \mathcode`j=\tlx@c \mathcode`k=\tlx@c \mathcode`l=\tlx@c
\mathcode`m=\tlx@c \mathcode`n=\tlx@c \mathcode`o=\tlx@c \mathcode`p=\tlx@c
\mathcode`q=\tlx@c \mathcode`r=\tlx@c \mathcode`s=\tlx@c \mathcode`t=\tlx@c
\mathcode`u=\tlx@c \mathcode`v=\tlx@c \mathcode`w=\tlx@c \mathcode`x=\tlx@c
\mathcode`y=\tlx@c \mathcode`z=\tlx@c
\c@tlx@ctr=\itfam \multiply\c@tlx@ctr"100\relax \advance\c@tlx@ctr "7041\relax
\mathcode`A=\tlx@c \mathcode`B=\tlx@c \mathcode`C=\tlx@c \mathcode`D=\tlx@c
\mathcode`E=\tlx@c \mathcode`F=\tlx@c \mathcode`G=\tlx@c \mathcode`H=\tlx@c
\mathcode`I=\tlx@c \mathcode`J=\tlx@c \mathcode`K=\tlx@c \mathcode`L=\tlx@c
\mathcode`M=\tlx@c \mathcode`N=\tlx@c \mathcode`O=\tlx@c \mathcode`P=\tlx@c
\mathcode`Q=\tlx@c \mathcode`R=\tlx@c \mathcode`S=\tlx@c \mathcode`T=\tlx@c
\mathcode`U=\tlx@c \mathcode`V=\tlx@c \mathcode`W=\tlx@c \mathcode`X=\tlx@c
\mathcode`Y=\tlx@c \mathcode`Z=\tlx@c
\makeatother

%%%%%%%%%%%%%%%%%%%%%%%%%%%%%%%%%%%%%%%%%%%%%%%%%%%%%%%%%%
%                THE describe ENVIRONMENT                %
%%%%%%%%%%%%%%%%%%%%%%%%%%%%%%%%%%%%%%%%%%%%%%%%%%%%%%%%%%
%
%
% It is like the description environment except it takes an argument
% ARG that should be the text of the widest label.  It adjusts the
% indentation so each item with label LABEL produces
%%      LABEL             blah blah blah
%%      <- width of ARG ->blah blah blah
%%                        blah blah blah
\newenvironment{describe}[1]%
   {\begin{list}{}{\settowidth{\labelwidth}{#1}%
            \setlength{\labelsep}{.5em}%
            \setlength{\leftmargin}{\labelwidth}% 
            \addtolength{\leftmargin}{\labelsep}%
            \addtolength{\leftmargin}{\parindent}%
            \def\makelabel##1{\rm ##1\hfill}}%
            \setlength{\topsep}{0pt}}%% 
                % Sets \topsep to 0 to reduce vertical space above
                % and below embedded displayed equations
   {\end{list}}

%   For tlatex.TeX
\usepackage{verbatim}
\makeatletter
\def\tla{\let\%\relax%
         \@bsphack
         \typeout{\%{\the\linewidth}}%
             \let\do\@makeother\dospecials\catcode`\^^M\active
             \let\verbatim@startline\relax
             \let\verbatim@addtoline\@gobble
             \let\verbatim@processline\relax
             \let\verbatim@finish\relax
             \verbatim@}
\let\endtla=\@esphack

\let\pcal=\tla
\let\endpcal=\endtla
\let\ppcal=\tla
\let\endppcal=\endtla

% The tlatex environment is used by TLATeX.TeX to typeset TLA+.
% TLATeX.TLA starts its files by writing a \tlatex command.  This
% command/environment sets \parindent to 0 and defines \% to its
% standard definition because the writing of the log files is messed up
% if \% is defined to be something else.  It also executes
% \@computerule to determine the dimensions for the TLA horizonatl
% bars.
\newenvironment{tlatex}{\@computerule%
                        \setlength{\parindent}{0pt}%
                       \makeatletter\chardef\%=`\%}{}


% The notla environment produces no output.  You can turn a 
% tla environment to a notla environment to prevent tlatex.TeX from
% re-formatting the environment.

\def\notla{\let\%\relax%
         \@bsphack
             \let\do\@makeother\dospecials\catcode`\^^M\active
             \let\verbatim@startline\relax
             \let\verbatim@addtoline\@gobble
             \let\verbatim@processline\relax
             \let\verbatim@finish\relax
             \verbatim@}
\let\endnotla=\@esphack

\let\nopcal=\notla
\let\endnopcal=\endnotla
\let\noppcal=\notla
\let\endnoppcal=\endnotla

%%%%%%%%%%%%%%%%%%%%%%%% end of tlatex.sty file %%%%%%%%%%%%%%%%%%%%%%% 
% last modified on Fri  3 August 2012 at 14:23:49 PST by lamport

\begin{document}
\tlatex
\@x{}\moduleLeftDash\@xx{ {\MODULE} Judgment}\moduleRightDash\@xx{}%
\@x{ {\EXTENDS} TLC ,\, FiniteSets ,\, Reals ,\, Sequences}%
\@pvspace{8.0pt}%
\@x{ {\VARIABLE} public ,\, private ,\, packages ,\, nextPackageId ,\,
    isReproducible}%
\@pvspace{8.0pt}%
\@x{}%
\@y{\@s{0}%
    How to run:
}%
\@xx{}%
\@x{}%
\@y{\@s{0}%
alias \ensuremath{tlc\.{=}}``\ensuremath{java \.{-}XX\.{:}\.{+}UseParallelGC
\.{-}Xmx12g \.{-}cp} \ensuremath{tla2tools.jar} \ensuremath{tlc2.TLC
\.{-}}workers auto''
}%
\@xx{}%
\@x{}%
\@y{\@s{0}%
\ensuremath{tlc \.{-}config} \ensuremath{Judgment3.cfg}
\ensuremath{Judgment3.tla
}}%
\@xx{}%
\@pvspace{16.0pt}%
\@x{ Nodes \.{\defeq} \{\@w{node1} ,\,\@w{node2} ,\,\@w{node3} \}}%
\@pvspace{8.0pt}%
\@x{ Packages \.{\defeq} \{\@w{package1} ,\,\@w{package2} ,\,\@w{package3}
,\,\@w{package3} \}}%
\@pvspace{8.0pt}%
\@x{ Status \.{\defeq} \{\@w{active} ,\,\@w{initialized} \}}%
\@pvspace{8.0pt}%
\@x{ IsReproducible ( p ) \.{\defeq} isReproducible [ p ]}%
\@pvspace{8.0pt}%
\@x{ Range ( f ) \.{\defeq} \{ f [ k ] \.{:} k \.{\in} {\DOMAIN} f \}}%
\@pvspace{8.0pt}%
\@x{ KeyValues ( m ) \.{\defeq} \{ {\langle} k ,\, m [ k ] {\rangle} \.{:} k
\.{\in} {\DOMAIN} m \}}%
\@pvspace{8.0pt}%
\@x{}%
\@y{\@s{0}%
Source: https:\ensuremath{\.{\slsl}learntla.com}/libraries/sets/
}%
\@xx{}%
\@x{ Pick ( S ) \.{\defeq} {\CHOOSE} s \.{\in} S \.{:} {\TRUE}}%
\@x{ {\RECURSIVE} SetReduce ( \_ ,\, \_ ,\, \_ )}%
\@x{ SetReduce ( Op ( \_ ,\, \_ ) ,\, S ,\, value ) \.{\defeq} {\IF} S \.{=}
\{ \} \.{\THEN} value}%
\@x{\@s{130.43} \.{\ELSE} \.{\LET} s \.{\defeq} Pick ( S )}%
\@x{\@s{130.43} \.{\IN} SetReduce ( Op ,\, S \.{\,\backslash\,} \{ s \} ,\,
Op ( s ,\, value ) )}%
\@pvspace{8.0pt}%
\@x{ Sum ( S ) \.{\defeq}}%
\@x{\@s{23.82} \.{\LET} \_op ( a ,\, b ) \.{\defeq} a \.{+} b}%
\@x{\@s{23.82} \.{\IN} SetReduce ( \_op ,\, S ,\, 0 )}%
\@pvspace{8.0pt}%
\@x{ InitialPrivate \.{\defeq} private \.{=} [}%
\@x{\@s{16.4} node1 \.{\mapsto} [}%
\@x{\@s{32.8} preferences \.{\mapsto} {\langle}}%
\@x{\@s{49.19} [ package \.{\mapsto}\@w{package1} ,\, level \.{\mapsto} 2 ,\,
status \.{\mapsto}\@w{not{-}processed} ] ,\,}%
\@x{\@s{49.19} [ package \.{\mapsto}\@w{package2} ,\, level \.{\mapsto} 2 ,\,
status \.{\mapsto}\@w{not{-}processed} ]}%
\@x{\@s{49.19}}%
\@y{\@s{0}%
[\ensuremath{package \.{\mapsto} \@w{package3}}, \ensuremath{level
\.{\mapsto} 2}, \ensuremath{status \.{\mapsto} \@w{not-processed}}]
}%
\@xx{}%
\@x{\@s{32.8} {\rangle}}%
\@x{\@s{16.4} ] ,\,}%
\@x{\@s{16.4} node2 \.{\mapsto} [}%
\@x{\@s{32.8} preferences \.{\mapsto} {\langle}}%
\@x{\@s{49.19}}%
\@y{\@s{0}%
[\ensuremath{package \.{\mapsto} \@w{package4}}, \ensuremath{level
\.{\mapsto} 2}, \ensuremath{status \.{\mapsto} \@w{not-processed}}],
}%
\@xx{}%
\@x{\@s{49.19} [ package \.{\mapsto}\@w{package3} ,\, level \.{\mapsto} 2 ,\,
status \.{\mapsto}\@w{not{-}processed} ] ,\,}%
\@x{\@s{49.19} [ package \.{\mapsto}\@w{package4} ,\, level \.{\mapsto} 2 ,\,
status \.{\mapsto}\@w{not{-}processed} ]}%
\@x{\@s{32.8} {\rangle}}%
\@x{\@s{16.4} ] ,\,}%
\@x{\@s{16.4} node3 \.{\mapsto} [}%
\@x{\@s{32.8} preferences \.{\mapsto} {\langle}}%
\@x{\@s{49.19} [ package \.{\mapsto}\@w{package5} ,\, level \.{\mapsto} 2 ,\,
status \.{\mapsto}\@w{not{-}processed} ] ,\,}%
\@x{\@s{49.19} [ package \.{\mapsto}\@w{package6} ,\, level \.{\mapsto} 2 ,\,
status \.{\mapsto}\@w{not{-}processed} ]}%
\@pvspace{8.0pt}%
\@x{\@s{32.8} {\rangle}}%
\@x{\@s{16.4} ]}%
\@x{}%
\@y{\@s{7.5}%
    ,
}%
\@xx{}%
\@x{}%
\@y{\@s{7.5}%
\ensuremath{node4 \.{\mapsto}} [
}%
\@xx{}%
\@x{}%
\@y{\@s{17.5}%
\ensuremath{preferences \.{\mapsto} {\langle}
}}%
\@xx{}%
\@x{}%
\@y{\@s{27.5}%
[\ensuremath{package \.{\mapsto}} ``\ensuremath{package7}'',
\ensuremath{level \.{\mapsto} 2}, \ensuremath{status \.{\mapsto}
\@w{not-processed}}],
}%
\@xx{}%
\@x{}%
\@y{\@s{27.5}%
[\ensuremath{package \.{\mapsto}} ``\ensuremath{package8}'',
\ensuremath{level \.{\mapsto} 2}, \ensuremath{status \.{\mapsto}
\@w{not-processed}}]
}%
\@xx{}%
\@x{}%
\@y{}%
\@xx{}%
\@x{}%
\@y{\@s{17.5}%
    \ensuremath{{\rangle}
    }}%
\@xx{}%
\@x{}%
\@y{\@s{7.5}%
    ]
}%
\@xx{}%
\@x{ ]}%
\@pvspace{8.0pt}%
\@x{ InitialPublic \.{\defeq} public \.{=} [}%
\@x{\@s{16.4} nodes \.{\mapsto} [}%
\@x{\@s{32.8} node1 \.{\mapsto} [ wallet \.{\mapsto} 3 ] ,\,}%
\@x{\@s{32.8} node2 \.{\mapsto} [ wallet \.{\mapsto} 0 ] ,\,}%
\@x{\@s{32.8} node3 \.{\mapsto} [ wallet \.{\mapsto} 0 ]}%
\@x{}%
\@y{\@s{17.5}%
    ,
}%
\@xx{}%
\@x{}%
\@y{\@s{17.5}%
\ensuremath{node4 \.{\mapsto}} [\ensuremath{wallet \.{\mapsto} 0}]
}%
\@xx{}%
\@x{\@s{16.4} ] ,\,}%
\@x{\@s{16.4} judgments \.{\mapsto} {\langle} {\rangle}}%
\@x{ ]}%
\@pvspace{8.0pt}%
\@x{ {\RECURSIVE} Cost ( \_ )}%
\@x{ Cost ( level ) \.{\defeq} {\IF} level \.{=} 0 \.{\THEN} 0 \.{\ELSE}
level \.{+} Cost ( level \.{-} 1 )}%
\@pvspace{8.0pt}%
\@x{ {\RECURSIVE} LevelHelp ( \_ ,\, \_ )}%
\@x{ LevelHelp ( available ,\, stepCost ) \.{\defeq}}%
\@x{\@s{16.4} {\IF} stepCost \.{>} available \.{\THEN} 0 \.{\ELSE} 1 \.{+}
LevelHelp ( available \.{-} stepCost ,\, stepCost \.{+} 1 )}%
\@pvspace{8.0pt}%
\@x{ Level ( cost ) \.{\defeq} LevelHelp ( cost ,\, 1 )}%
\@pvspace{8.0pt}%
\@x{ {\RECURSIVE} WalletUpdatesForVoters ( \_ ,\, \_ ,\, \_ ,\, \_ )}%
\@x{ WalletUpdatesForVoters ( judgment ,\, votes ,\, reward ,\, updates )
\.{\defeq}}%
\@x{\@s{16.4} \.{\LET} finalResult \.{\defeq} ToString ( judgment . tally .
for \.{>} judgment . tally . against )}%
\@x{\@s{36.79} judge \.{\defeq} Head ( votes ) . judge}%
\@x{\@s{36.79} vote \.{\defeq} Head ( votes ) . vote}%
\@x{\@s{16.4} \.{\IN}}%
\@x{\@s{32.8} {\IF} Len ( votes ) \.{=} 0}%
\@x{\@s{49.05} \.{\THEN} updates}%
\@x{\@s{32.8} \.{\ELSE} {\IF} judge \.{=} judgment . owner}%
\@x{\@s{49.19}}%
\@y{\@s{0}%
    Skip the owner
}%
\@xx{}%
\@x{\@s{49.19} \.{\THEN} WalletUpdatesForVoters ( judgment ,\, Tail ( votes )
,\, reward ,\, updates )}%
\@x{\@s{32.8} \.{\ELSE} {\IF} vote \.{\neq} finalResult \.{\THEN} \.{\LET}}%
\@x{\@s{49.19}}%
\@y{\@s{0}%
    Minority voter
}%
\@xx{}%
\@x{\@s{65.6} newUpdates \.{\defeq} [ updates {\EXCEPT} {\bang} [ judge ] .
wallet \.{=} @ ]}%
\@x{\@s{49.19} \.{\IN} WalletUpdatesForVoters ( judgment ,\, Tail ( votes )
,\, reward ,\, newUpdates )}%
\@x{\@s{32.8} \.{\ELSE} \.{\LET}}%
\@x{\@s{49.19}}%
\@y{\@s{0}%
    Majority voter
}%
\@xx{}%
\@x{\@s{65.6} newUpdates \.{\defeq} [ updates {\EXCEPT} {\bang} [ judge ] .
wallet \.{=} @ \.{+} reward \.{+} 1 ]}%
\@x{\@s{73.69} newReward \.{\defeq} {\IF} reward \.{=} 0 \.{\THEN} 0
\.{\ELSE} reward \.{-} 1}%
\@x{\@s{53.29} \.{\IN} WalletUpdatesForVoters ( judgment ,\, Tail ( votes )
,\, newReward ,\, newUpdates )}%
\@pvspace{24.0pt}%
\@x{ {\RECURSIVE} WalletUpdatesFromVotes ( \_ ,\, \_ ,\, \_ )}%
\@x{ WalletUpdatesFromVotes ( judgment ,\, votes ,\, updates ) \.{\defeq}}%
\@x{\@s{16.4} {\IF} Len ( votes ) \.{=} 0}%
\@x{\@s{32.65} \.{\THEN} updates}%
\@x{\@s{32.65} \.{\ELSE} \.{\LET}}%
\@x{\@s{76.26} finalResult \.{\defeq} ToString ( judgment . tally . for \.{>}
judgment . tally . against )}%
\@x{\@s{76.26} judge \.{\defeq} Head ( votes ) . judge}%
\@x{\@s{76.26}}%
\@y{\@s{0}%
We add \mbox{'}\ensuremath{1\.{'}} for free (but not to the owner)
}%
\@xx{}%
\@x{\@s{76.26} participationReward \.{\defeq} {\IF} judge \.{=} judgment .
owner \.{\THEN} 0 \.{\ELSE} 1}%
\@x{\@s{76.26} positionalReward \.{\defeq} Cardinality ( \{ v \.{\in} Range (
votes ) \.{:} v . vote \.{=} finalResult \} )}%
\@x{\@s{76.26} reward \.{\defeq} {\IF} Head ( votes ) . vote \.{=}
finalResult}%
\@x{\@s{133.26} \.{\THEN} participationReward \.{+} positionalReward}%
\@x{\@s{133.26} \.{\ELSE} 0}%
\@x{\@s{76.26} newUpdates \.{\defeq} [ updates {\EXCEPT} {\bang} [ judge ] .
wallet \.{=} @ \.{+} reward ]}%
\@x{\@s{49.05} \.{\IN} WalletUpdatesFromVotes ( judgment ,\, Tail ( votes )
,\, newUpdates )}%
\@pvspace{16.0pt}%
\@x{ {\RECURSIVE} WalletUpdatesForNotVoting ( \_ ,\, \_ )}%
\@x{ WalletUpdatesForNotVoting ( nonVoters ,\, updates ) \.{\defeq}}%
\@x{\@s{16.4} {\IF} nonVoters \.{=} \{ \}}%
\@x{\@s{32.65} \.{\THEN} updates}%
\@x{\@s{32.65} \.{\ELSE}}%
\@x{\@s{49.05} \.{\LET} v \.{\defeq} {\CHOOSE} voter \.{\in} nonVoters \.{:}
{\TRUE}}%
\@x{\@s{73.55} newUpdates \.{\defeq} [ updates {\EXCEPT} {\bang} [ v ] .
wallet \.{=} @ \.{-} 1 ]}%
\@x{\@s{49.05} \.{\IN}\@s{4.09} WalletUpdatesForNotVoting ( nonVoters
\.{\,\backslash\,} \{ v \} ,\, newUpdates )}%
\@pvspace{8.0pt}%
\@x{}%
\@y{\@s{0}%
    Base reward on timing of open vote
}%
\@xx{}%
\@x{ Rewards ( judgment ) \.{\defeq}}%
\@x{\@s{16.4} \.{\LET} secretVoters \.{\defeq} \{ j . judge \.{:} j \.{\in}
Range ( judgment . secretJudgments ) \}}%
\@x{\@s{36.79} openVotersWithoutOwner \.{\defeq} \{ j \.{\in} Range (
judgment . openJudgments ) \.{:} j . judge \.{\neq} judgment . owner \}}%
\@x{\@s{36.79} nonVoters \.{\defeq} Nodes \.{\,\backslash\,} secretVoters}%
\@x{\@s{36.79} cost \.{\defeq} Cost ( Cardinality ( openVotersWithoutOwner )
)}%
\@x{\@s{36.79} updatesForVoters \.{\defeq} WalletUpdatesFromVotes ( judgment
,\, judgment . openJudgments ,\, [ n \.{\in} Nodes \.{\mapsto} public .
nodes [ n ] ] )}%
\@x{\@s{36.79} updatesForNonVoters \.{\defeq} WalletUpdatesForNotVoting (
nonVoters ,\, updatesForVoters )}%
\@x{\@s{40.89}}%
\@y{\@s{0}%
    Owners have to pay the cost
}%
\@xx{}%
\@x{\@s{16.4} \.{\IN}\@s{4.09} [ updatesForNonVoters {\EXCEPT} {\bang} [
        judgment . owner ] . wallet \.{=} @ \.{-} cost ]}%
\@pvspace{16.0pt}%
\@x{ FutureCost ( node ) \.{\defeq}}%
\@x{\@s{16.4} \.{\LET} NotProcessedYet \.{\defeq} \{ p \.{\in} Range (
private [ node ] . preferences ) \.{:} p . status \.{\neq}\@w{processed} \}}%
\@x{\@s{16.4} \.{\IN} Sum ( \{ Cost ( p . level ) \.{:} p \.{\in}
NotProcessedYet \} )}%
\@pvspace{16.0pt}%
\@x{ AllPreferences \.{\defeq}}%
\@x{\@s{16.4} \.{\LET} op ( a ,\, b ) \.{\defeq} a \.{\circ} b}%
\@x{\@s{16.4} \.{\IN} SetReduce ( op ,\, \{ p . preferences \.{:} p \.{\in}
Range ( private ) \} ,\, {\langle} {\rangle} )}%
\@pvspace{8.0pt}%
\@x{ Min ( S ) \.{\defeq}}%
\@x{\@s{22.31} {\CHOOSE} e \.{\in} S \.{\cup} \{ 0 \} \.{:}}%
\@x{\@s{38.71} \A\, e2 \.{\in} S \.{:} e \.{\leq} e2}%
\@pvspace{8.0pt}%
\@x{ SecretJudges ( judgment ) \.{\defeq}}%
\@x{\@s{16.4} \{ j . judge \.{:} j \.{\in} Range ( judgment . secretJudgments
) \}}%
\@pvspace{8.0pt}%
\@x{ OpenJudges ( judgment ) \.{\defeq}}%
\@x{\@s{16.4} \{ j . judge \.{:} j \.{\in} Range ( judgment . openJudgments )
\}}%
\@pvspace{8.0pt}%
\@x{ Rewards2 ( judgment ) \.{\defeq}}%
\@x{\@s{16.4} \.{\LET}\@s{5.06} openVotes\@s{1.99} \.{\defeq} judgment .
openJudgments}%
\@x{\@s{41.86} topReward \.{\defeq} judgment . targetVotes}%
\@x{\@s{41.86} wallets \.{\defeq} public . nodes}%
\@x{\@s{41.86} openVoters \.{\defeq} OpenJudges ( judgment )}%
\@x{\@s{41.86} nonVoters\@s{1.63} \.{\defeq} Nodes \.{\,\backslash\,}
openVoters}%
\@x{\@s{41.86} ownerCost \.{\defeq} Cost ( judgment . targetVotes )}%
\@x{\@s{41.86} updatesForVoters \.{\defeq} WalletUpdatesForVoters ( judgment
,\, openVotes ,\, topReward ,\, wallets )}%
\@x{\@s{41.86} updatesForNonVoters \.{\defeq} WalletUpdatesForNotVoting (
nonVoters ,\, updatesForVoters )}%
\@x{\@s{16.4} \.{\IN}\@s{5.06} [ updatesForNonVoters {\EXCEPT} {\bang} [
        judgment . owner ] . wallet \.{=} @ \.{-} ownerCost ]}%
\@pvspace{24.0pt}%
\@x{ AddPreferencedPackage ( node ) \.{\defeq}}%
\@x{\@s{16.4} \.{\LET} currentBalance \.{\defeq} public . nodes [ node ] .
wallet}%
\@x{\@s{36.79} pendingPackagesCount \.{\defeq} Level ( FutureCost ( node ) )}%
\@x{\@s{36.79} maxJudgmentCost \.{\defeq} currentBalance \.{-} FutureCost (
node )}%
\@x{\@s{36.79} maxJudgmentLevel \.{\defeq} Level ( maxJudgmentCost )}%
\@x{\@s{36.79} packageLevel \.{\defeq} maxJudgmentLevel}%
\@x{\@s{36.79} packageName \.{\defeq}\@w{package} \.{\circ} ToString (
nextPackageId )}%
\@x{\@s{36.79} package \.{\defeq} [ package \.{\mapsto} packageName ,\,}%
\@x{\@s{92.43} level \.{\mapsto} packageLevel ,\,}%
\@x{\@s{92.43} status \.{\mapsto}\@w{not{-}processed} ]}%
\@x{\@s{16.4} \.{\IN}}%
\@x{\@s{32.8} \.{\land} packageLevel \.{>} 1}%
\@x{\@s{32.8} \.{\land} pendingPackagesCount \.{<} 2}%
\@x{\@s{32.8} \.{\land} private \.{'} \.{=} [ private {\EXCEPT} {\bang} [
        node ] . preferences \.{=} Append ( @ ,\, package ) ]}%
\@x{}%
\@y{\@s{17.5}%
\ensuremath{\.{\land} isReproducible\.{'} \.{=}}
[\ensuremath{isReproducible} \ensuremath{{\EXCEPT} {\bang}[packageName]
\.{=} {\TRUE}}]
}%
\@xx{}%
\@x{\@s{32.8} \.{\land} nextPackageId \.{'} \.{=} nextPackageId \.{+} 1}%
\@pvspace{8.0pt}%
\@x{ NoActiveJudgments \.{\defeq}}%
\@x{\@s{16.4} \A\, j \.{\in} Range ( public . judgments ) \.{:} j . status
\.{\neq}\@w{active}}%
\@pvspace{8.0pt}%
\@x{ RunSimulation \.{\defeq} Cardinality ( \{ j \.{\in} Range ( public .
judgments ) \.{:} j . status \.{=}\@w{closed} \} ) \.{<} 4}%
\@pvspace{8.0pt}%
\@x{}%
\@y{\@s{0}%
    \ceqdash{8} \ensuremath{Chaincode} \ceqdash{8}
}%
\@xx{}%
\@pvspace{8.0pt}%
\@x{ InitializeJudgment ( package ,\, owner ,\, targetVotes ) \.{\defeq}}%
\@x{\@s{16.4}}%
\@y{\@s{0}%
    Guards
}%
\@xx{}%
\@x{}%
\@y{\@s{7.5}%
\ensuremath{\.{\land} NoActiveJudgments \.{\,\backslash\,}\.{*}} This is
*optional*
}%
\@xx{}%
\@x{\@s{16.4} \.{\land} Cost ( targetVotes ) \.{\leq} public . nodes [ owner
    ] . wallet}%
\@x{\@s{16.4} \.{\land} ( \A\, j \.{\in} Range ( public . judgments ) \.{:} j
. status \.{=}\@w{active} \.{\implies} j . owner \.{\neq} owner )}%
\@x{\@s{16.4}}%
\@y{\@s{0}%
    Update
}%
\@xx{}%
\@x{\@s{16.4}}%
\@y{\@s{0}%
    \ensuremath{TODO}: Better ids. Currently there can only be one judgment per
    package.
}%
\@xx{}%
\@x{\@s{16.4} \.{\land} \.{\LET} judgmentId \.{\defeq} package \.{\IN}}%
\@x{\@s{31.61} public \.{'} \.{=} [ public {\EXCEPT} {\bang} . judgments
\.{=} Append ( @ ,\,}%
\@x{\@s{80.49} [\@s{8.2} id \.{\mapsto} judgmentId ,\,}%
\@x{\@s{91.47} tally \.{\mapsto} [ for \.{\mapsto} 0 ,\, against \.{\mapsto}
0 ] ,\,}%
\@x{\@s{91.47} finalResult \.{\mapsto}\@w{undecided} ,\,}%
\@x{\@s{91.47} package \.{\mapsto} package ,\,}%
\@x{\@s{91.47} owner \.{\mapsto} owner ,\,}%
\@x{\@s{91.47} targetVotes \.{\mapsto} targetVotes ,\,}%
\@x{\@s{91.47} status \.{\mapsto}\@w{active} ,\,}%
\@x{\@s{91.47} phase \.{\mapsto}\@w{secretVotes} ,\,}%
\@x{\@s{91.47} secretJudgments \.{\mapsto} {\langle} {\rangle} ,\,}%
\@x{\@s{91.47} openJudgments \.{\mapsto} {\langle} {\rangle} ] ) ]}%
\@pvspace{8.0pt}%
\@x{ AddSecretJudgment ( judgmentId ,\, judge ,\, secretVote ) \.{\defeq}}%
\@x{\@s{16.4} \.{\lor}\@s{4.1} Len ( public . judgments ) \.{=} 0}%
\@x{\@s{16.4} \.{\lor}\@s{4.09} \E\, index \.{\in} {\DOMAIN} public .
judgments \.{:}}%
\@x{\@s{42.93} \.{\LET} j \.{\defeq} public . judgments [ index ] \.{\IN}}%
\@x{\@s{63.33} \.{\land}\@s{4.1} j . id \.{=} judgmentId}%
\@x{\@s{63.33} \.{\land} j . status \.{=}\@w{active}}%
\@x{\@s{63.33} \.{\land} j . phase \.{=}\@w{secretVotes}}%
\@x{\@s{63.33}}%
\@y{\@s{0}%
Should be able to create a majority with \ensuremath{j.targetVotes
\.{\neq}votes
}}%
\@xx{}%
\@x{}%
\@y{\@s{37.5}%
\ensuremath{\.{\land} Len(j.secretJudgments) \.{+} 1 \.{<} 2 \.{*}
j.targetVotes
}}%
\@xx{}%
\@x{\@s{65.6}}%
\@y{\@s{0}%
Has \mbox{'}judge\mbox{'} added their vote before\.{?}
}%
\@xx{}%
\@x{\@s{65.6} \.{\land} \{ sj \.{\in} Range ( j . secretJudgments ) \.{:} sj
. judge \.{=} judge \} \.{=} \{ \}}%
\@x{\@s{65.6}}%
\@y{\@s{0}%
    Update
}%
\@xx{}%
\@x{\@s{65.6} \.{\land}\@s{4.1} public \.{'} \.{=} [ public {\EXCEPT} {\bang}
. judgments \.{=} [ @ {\EXCEPT} {\bang} [ index ] \.{=}}%
\@x{\@s{80.81} [ @ {\EXCEPT} {\bang} . secretJudgments \.{=} Append ( @ ,\,}%
\@x{\@s{95.46} [\@s{8.2} judge \.{\mapsto} judge ,\,}%
\@x{\@s{106.44} vote \.{\mapsto} secretVote ] ) ] ] ]}%
\@pvspace{8.0pt}%
\@x{ EndSecretSubmissions ( judgmentId ,\, owner ) \.{\defeq}}%
\@x{\@s{16.4} \.{\lor} Len ( public . judgments ) \.{=} 0}%
\@x{\@s{16.4} \.{\lor} \E\, index \.{\in} {\DOMAIN} public . judgments \.{:}}%
\@x{\@s{31.61} \.{\LET} j \.{\defeq} public . judgments [ index ] \.{\IN}}%
\@x{\@s{52.01} \.{\land} j . id \.{=} judgmentId}%
\@x{\@s{52.01} \.{\land} j . status \.{=}\@w{active}}%
\@x{\@s{52.01} \.{\land} j . phase\@s{2.49} \.{=}\@w{secretVotes}}%
\@x{\@s{52.01} \.{\land} j . owner \.{=} owner}%
\@x{\@s{52.01}}%
\@y{\@s{0}%
Should this be in the smart contract\.{?}
}%
\@xx{}%
\@x{\@s{52.01} \.{\land} 2 \.{*} j . targetVotes \.{-} 1 \.{\leq} Len ( j .
secretJudgments )}%
\@x{\@s{52.01} \.{\land} public \.{'} \.{=} [ public {\EXCEPT} {\bang} .
judgments \.{=} [ @ {\EXCEPT} {\bang} [ index ] \.{=}}%
\@x{\@s{67.22} [ @ {\EXCEPT} {\bang} . phase \.{=}\@w{openVotes} ] ] ]}%
\@pvspace{8.0pt}%
\@x{ ShowJudgment ( judgmentId ,\, judge ,\, openVote ) \.{\defeq}}%
\@x{\@s{16.4} \.{\lor}\@s{4.1} Len ( public . judgments ) \.{=} 0}%
\@x{\@s{16.4} \.{\lor}\@s{4.09} \E\, index \.{\in} {\DOMAIN} public .
judgments \.{:}}%
\@x{\@s{42.93} \.{\LET} j \.{\defeq} public . judgments [ index ] \.{\IN}}%
\@x{\@s{63.33} \.{\land} j . id \.{=} judgmentId}%
\@x{\@s{63.33} \.{\land} j . status \.{=}\@w{active}}%
\@x{\@s{63.33} \.{\land} j . phase \.{=}\@w{openVotes}}%
\@x{\@s{63.33}}%
\@y{\@s{0}%
Has \mbox{'}judge\mbox{'} added their vote before\.{?}
}%
\@xx{}%
\@x{\@s{63.33} \.{\land} \{ sj \.{\in} Range ( j . secretJudgments ) \.{:} sj
. judge \.{=} judge \} \.{\neq} \{ \}}%
\@x{\@s{63.33}}%
\@y{\@s{0}%
Has \mbox{'}judge\mbox{'} showed their vote before\.{?}
}%
\@xx{}%
\@x{\@s{63.33} \.{\land} \{ oj \.{\in} Range ( j . openJudgments ) \.{:} oj .
judge \.{=} judge \} \.{=} \{ \}}%
\@x{\@s{63.33}}%
\@y{\@s{0}%
    Secret and open votes should be the same
}%
\@xx{}%
\@x{\@s{63.33}}%
\@y{\@s{0}%
    This validation is done with \ensuremath{HMAC} in practice
}%
\@xx{}%
\@x{\@s{63.33} \.{\land} \E\, sj \.{\in} Range ( j . secretJudgments ) \.{:}}%
\@x{\@s{78.54} \.{\land}\@s{4.1} sj . judge\@s{5.50} \.{=} judge}%
\@x{\@s{78.54} \.{\land}\@s{4.09} openVote \.{=} sj . vote}%
\@x{\@s{63.33}}%
\@y{\@s{0}%
    No majority yet!
}%
\@xx{}%
\@x{}%
\@y{\@s{37.5}%
\ensuremath{\.{\land} 2 \.{*} j.tally.for \.{\leq} Len(j.secretJudgments)
}}%
\@xx{}%
\@x{}%
\@y{\@s{37.5}%
\ensuremath{\.{\land} 2 \.{*} j.tally.against \.{\leq} Len(j.secretJudgments)
}}%
\@xx{}%
\@x{\@s{65.6}}%
\@y{\@s{0}%
    Update
}%
\@xx{}%
\@x{\@s{65.6} \.{\land}\@s{4.1} public \.{'} \.{=} [ public {\EXCEPT} {\bang}
. judgments [ index ] \.{=} [ @ {\EXCEPT}}%
\@x{\@s{129.69} {\bang} . openJudgments \.{=} Append ( @ ,\,}%
\@x{\@s{145.23} [\@s{8.2} judge \.{\mapsto} judge ,\,}%
\@x{\@s{156.21} vote \.{\mapsto} openVote ] ) ,\,}%
\@x{\@s{129.69} {\bang} . tally . for \.{=} {\IF} openVote \.{=}\@w{TRUE}
\.{\THEN} @ \.{+} 1 \.{\ELSE} @ ,\,}%
\@x{\@s{129.69} {\bang} . tally . against \.{=} {\IF} openVote
\.{=}\@w{FALSE} \.{\THEN} @ \.{+} 1 \.{\ELSE} @ ] ]}%
\@pvspace{16.0pt}%
\@x{ CloseJudgment ( judgmentId ,\, owner ) \.{\defeq}}%
\@x{\@s{16.4} \.{\lor}\@s{4.1} Len ( public . judgments ) \.{=} 0}%
\@x{\@s{16.4} \.{\lor}\@s{4.09} \E\, index \.{\in} {\DOMAIN} public .
judgments \.{:}}%
\@x{\@s{42.93} \.{\LET} j \.{\defeq} public . judgments [ index ]}%
\@x{\@s{63.33} finalResult \.{\defeq} ToString ( j . tally . for \.{>} j .
tally . against ) \.{\IN}}%
\@x{}%
\@y{\@s{37.5}%
\ensuremath{\.{\land} PrintT(j.id \.{\circ} j.status \.{\circ} j.phase
\.{\circ} j.owner)
}}%
\@xx{}%
\@x{\@s{65.6} \.{\land}\@s{4.1} j . id \.{=} judgmentId}%
\@x{\@s{65.6} \.{\land}\@s{4.09} j . status \.{=}\@w{active}}%
\@x{\@s{65.6} \.{\land}\@s{4.09} j . phase\@s{2.49} \.{=}\@w{openVotes}}%
\@x{\@s{65.6} \.{\land}\@s{4.09} j . owner \.{=} owner}%
\@x{\@s{80.81}}%
\@y{\@s{0}%
    Undecided result, majority FOR, or majority AGAINST
}%
\@xx{}%
\@x{}%
\@y{\@s{37.5}%
\ensuremath{\.{\land} \.{\lor} Len(j.openJudgments) \.{=}
Len(j.secretJudgments)
}}%
\@xx{}%
\@x{}%
\@y{\@s{47.5}%
\ensuremath{\.{\lor} 2 \.{*} j.tally.for \.{>} Len(j.secretJudgments)
}}%
\@xx{}%
\@x{}%
\@y{\@s{47.5}%
\ensuremath{\.{\lor} 2 \.{*} j.tally.against \.{>} Len(j.secretJudgments)
}}%
\@xx{}%
\@x{\@s{82.0}}%
\@y{\@s{0}%
    We have a majority \ensuremath{for{\bang}
    }}%
\@xx{}%
\@x{\@s{65.6} \.{\land}\@s{5.28} \.{\lor}\@s{4.1} \.{\land}\@s{4.1} j . tally
. for \.{>} j . tally . against}%
\@x{\@s{97.21} \.{\land}\@s{4.09} public \.{'} \.{=} [ public {\EXCEPT}}%
\@x{\@s{128.82} {\bang} . judgments \.{=} [ @ {\EXCEPT} {\bang} [ index ]
\.{=}}%
\@x{\@s{144.36} [ @ {\EXCEPT}}%
\@x{\@s{159.02} {\bang} . status \.{=}\@w{closed} ,\,}%
\@x{\@s{159.02} {\bang} . phase \.{=}\@w{judgment} ,\,}%
\@x{\@s{159.02} {\bang} . finalResult \.{=}\@w{TRUE} ] ] ,\,}%
\@x{\@s{128.82} {\bang} . nodes \.{=} Rewards2 ( j ) ]}%
\@x{\@s{82.0}}%
\@y{\@s{0}%
    We have a majority \ensuremath{against{\bang}
    }}%
\@xx{}%
\@x{\@s{82.0} \.{\lor}\@s{4.1} \.{\land}\@s{4.1} j . tally . against \.{>} j
. tally . for}%
\@x{\@s{97.21} \.{\land}\@s{4.09} public \.{'} \.{=} [ public {\EXCEPT}}%
\@x{\@s{128.82} {\bang} . judgments \.{=} [ @ {\EXCEPT} {\bang} [ index ]
\.{=}}%
\@x{\@s{144.36} [ @ {\EXCEPT}}%
\@x{\@s{159.02} {\bang} . status \.{=}\@w{closed} ,\,}%
\@x{\@s{159.02} {\bang} . phase \.{=}\@w{judgment} ,\,}%
\@x{\@s{159.02} {\bang} . finalResult \.{=}\@w{FALSE} ] ] ,\,}%
\@x{\@s{128.82} {\bang} . nodes \.{=} Rewards2 ( j ) ]}%
\@x{\@s{82.0}}%
\@y{\@s{0}%
    Everyone has voted, and we don\mbox{'}t have a majority
}%
\@xx{}%
\@x{\@s{82.0} \.{\lor}\@s{4.1} \.{\land}\@s{4.1} j . tally . for \.{=} j .
tally . against}%
\@x{\@s{97.21} \.{\land}\@s{4.09} public \.{'} \.{=} [ public {\EXCEPT}
    {\bang} . judgments [ index ] \.{=} [ @ {\EXCEPT}}%
\@x{\@s{141.09} {\bang} . status \.{=}\@w{closed} ,\,}%
\@x{\@s{141.09} {\bang} . finalResult \.{=}\@w{undecided} ] ]}%
\@x{\@s{82.0}}%
\@y{\@s{0}%
\ensuremath{\.{\land} PrintT(\mbox{``}Majority\.{:} \mbox{''} \.{\circ}
ToString(public\.{'}.judgments[index].tally.for) \.{\circ} \mbox{``} to
\mbox{''} \.{\circ} ToString(public\.{'}.judgments[index].tally.against))
}}%
\@xx{}%
\@pvspace{16.0pt}%
\@x{}%
\@y{\@s{0}%
    \ceqdash{8} Invariants \ceqdash{8}
}%
\@xx{}%
\@pvspace{8.0pt}%
\@x{ OwnOneJudgmentAtATime \.{\defeq}}%
\@x{\@s{16.4}}%
\@y{\@s{0}%
For all active judgments \ensuremath{j}, \ensuremath{j2} where \ensuremath{j
\.{\neq} j2}: \ensuremath{j.owner \.{\neq} j2.owner
}}%
\@xx{}%
\@x{\@s{16.4}}%
\@y{\@s{0}%
    Needed to ensure that judgments can be paid for.
}%
\@xx{}%
\@x{\@s{16.4} \A\, j \.{\in} Range ( public . judgments ) \.{:} j . status
\.{=}\@w{active} \.{\implies}}%
\@x{\@s{27.72} ( \A\, j2 \.{\in} Range ( public . judgments ) \.{:} j2 .
status \.{=}\@w{active} \.{\land}}%
\@x{\@s{38.83} j . id \.{\neq} j2 . id \.{\implies} j . owner \.{\neq} j2 .
owner )}%
\@pvspace{8.0pt}%
\@x{ LessActiveThenNodes \.{\defeq}}%
\@x{\@s{16.4}}%
\@y{\@s{0}%
    There should never be more active judgments than nodes
}%
\@xx{}%
\@x{\@s{16.4} \.{\LET} active \.{\defeq} \{ j \.{\in} Range ( public .
judgments ) \.{:} j . status \.{=}\@w{active} \}}%
\@x{\@s{16.4} \.{\IN} Cardinality ( active ) \.{\leq} Cardinality ( Nodes )}%
\@pvspace{8.0pt}%
\@x{}%
\@y{\@s{0}%
    \ensuremath{TODO}: Add filtering so this one works
}%
\@xx{}%
\@x{ CostLessThanMaxCost \.{\defeq}}%
\@x{\@s{16.4} \.{\lor}\@s{4.1} Len ( public . judgments ) \.{=} 0}%
\@x{\@s{16.4} \.{\lor}\@s{4.09} \A\, j \.{\in} Range ( public . judgments )
\.{:} Len ( j . secretJudgments ) \.{\leq} j . targetVotes}%
\@pvspace{8.0pt}%
\@x{ AllButOnePackagedIsJudgedPerNode \.{\defeq} \A\, n \.{\in} Nodes \.{:}}%
\@x{\@s{32.8} Cardinality ( \{ p \.{\in} Range ( private [ n ] . preferences
) \.{:} p . status \.{\neq}\@w{processed} \} ) \.{\leq} 1}%
\@pvspace{8.0pt}%
\@x{ TooFewWantsToBuild \.{\defeq}}%
\@x{\@s{16.4} \.{\LET}\@s{16.4} WantsToBuild \.{\defeq} \{ n \.{\in} Nodes
\.{:} FutureCost ( n ) \.{>} public . nodes [ n ] . wallet \.{-} 1 \}}%
\@x{\@s{53.19} NotYetProcessed \.{\defeq} \{ p \.{\in} Range ( AllPreferences
) \.{:} p . status \.{=}\@w{not{-}processed} \}}%
\@x{\@s{53.19} MinimumBuilderDemand \.{\defeq} Min ( \{ p . level \.{:} p
\.{\in} NotYetProcessed \} )}%
\@x{\@s{16.4} \.{\IN}\@s{4.1} Cardinality ( WantsToBuild ) \.{<} 2 \.{*}
MinimumBuilderDemand}%
\@pvspace{16.0pt}%
\@x{ NoPackagesAreWronglyJudged \.{\defeq}}%
\@x{\@s{16.4} \A\, j \.{\in} Range ( public . judgments ) \.{:}}%
\@x{\@s{27.72} \.{\lor}\@s{4.1} j . finalResult \.{=} IsReproducible ( j . id
)}%
\@x{\@s{27.72} \.{\lor}\@s{4.09} j . finalResult \.{=}\@w{undecided}}%
\@pvspace{16.0pt}%
\@x{ Fairness \.{\defeq} {\WF}_{ public} ( AllButOnePackagedIsJudgedPerNode
)}%
\@pvspace{8.0pt}%
\@x{}%
\@y{\@s{0}%
    \ceqdash{8} \ensuremath{Spec} \ceqdash{8}
}%
\@xx{}%
\@pvspace{8.0pt}%
\@x{ Init \.{\defeq}}%
\@x{\@s{16.4} \.{\land} InitialPublic}%
\@x{\@s{16.4} \.{\land} InitialPrivate}%
\@x{\@s{16.4} \.{\land} packages \.{=} Packages}%
\@x{\@s{16.4} \.{\land} nextPackageId \.{=} 9}%
\@x{\@s{16.4} \.{\land} isReproducible \.{=} [ package1 \.{\mapsto}\@w{TRUE}
,\,}%
\@x{\@s{106.22} package2 \.{\mapsto}\@w{FALSE} ,\,}%
\@x{\@s{106.22} package3 \.{\mapsto}\@w{TRUE} ,\,}%
\@x{\@s{106.22} package4 \.{\mapsto}\@w{TRUE} ,\,}%
\@x{\@s{106.22} package5 \.{\mapsto}\@w{TRUE} ,\,}%
\@x{\@s{106.22} package6 \.{\mapsto}\@w{TRUE} ,\,}%
\@x{\@s{106.22} package7 \.{\mapsto}\@w{TRUE} ,\,}%
\@x{\@s{106.22} package8 \.{\mapsto}\@w{TRUE} ,\,}%
\@x{\@s{106.22} package9 \.{\mapsto}\@w{FALSE} ,\,}%
\@x{\@s{106.22} package10 \.{\mapsto}\@w{FALSE} ,\,}%
\@x{\@s{106.22} package11 \.{\mapsto}\@w{FALSE} ,\,}%
\@x{\@s{106.22} package12 \.{\mapsto}\@w{FALSE} ,\,}%
\@x{\@s{106.22} package13 \.{\mapsto}\@w{FALSE} ,\,}%
\@x{\@s{106.22} package14 \.{\mapsto}\@w{FALSE} ,\,}%
\@x{\@s{106.22} package15 \.{\mapsto}\@w{FALSE} ,\,}%
\@x{\@s{106.22} package16 \.{\mapsto}\@w{FALSE} ,\,}%
\@x{\@s{106.22} package17 \.{\mapsto}\@w{FALSE} ,\,}%
\@x{\@s{106.22} package18 \.{\mapsto}\@w{FALSE} ,\,}%
\@x{\@s{106.22} package19 \.{\mapsto}\@w{FALSE} ,\,}%
\@x{\@s{106.22} package20 \.{\mapsto}\@w{FALSE} ]}%
\@pvspace{8.0pt}%
\@x{ Next \.{\defeq}}%
\@x{}%
\@y{\@s{7.5}%
\ensuremath{\.{\land} {\IF}} \ensuremath{Cardinality(\{j \.{\in}
Range(public.judgments)\.{:} j.status \.{=} \@w{closed}\}) \.{\neq} 0}
\ensuremath{\.{\THEN} PrintT(Cardinality(\{j \.{\in}
Range(public.judgments)\.{:} j.status \.{=} \@w{closed}\})) \.{\ELSE} {\TRUE}
}}%
\@xx{}%
\@x{}%
\@y{\@s{7.5}%
\ensuremath{\.{\land} Cardinality(\{j \.{\in} Range(public.judgments)\.{:}
j.status \.{=} \@w{closed}\}) \.{<} 1
}}%
\@xx{}%
\@x{\@s{32.8} \.{\lor}\@s{4.1} \.{\land}\@s{4.1}
AllButOnePackagedIsJudgedPerNode}%
\@x{\@s{48.01} \.{\land}\@s{4.10} RunSimulation}%
\@x{\@s{48.01} \.{\land} {\UNCHANGED} {\langle} public ,\, private ,\,
packages ,\, nextPackageId ,\, isReproducible {\rangle}}%
\@x{\@s{16.4}}%
\@y{\@s{7.5}%
\ensuremath{\.{\lor} \.{\land} TooFewWantsToBuild
}}%
\@xx{}%
\@x{\@s{16.4}}%
\@y{\@s{17.5}%
\ensuremath{\.{\land} {\UNCHANGED} {\langle}public,\, private,\, packages,\,
nextPackageId,\, isReproducible{\rangle}
}}%
\@xx{}%
\@x{\@s{32.8} \.{\lor}\@s{4.1} \.{\land}\@s{4.1} Cardinality ( \{ j \.{\in}
Range ( public . judgments ) \.{:} j . status \.{=}\@w{processed} \} ) \.{=}
1}%
\@x{\@s{48.01} \.{\land}\@s{4.10} RunSimulation}%
\@x{\@s{48.01} \.{\land} {\UNCHANGED} {\langle} public ,\, private ,\,
packages ,\, nextPackageId ,\, isReproducible {\rangle}}%
\@x{\@s{32.8} \.{\lor}\@s{4.09} \E\, n \.{\in} Nodes \.{:}}%
\@x{\@s{48.01} \E\, index \.{\in} {\DOMAIN} private [ n ] . preferences
\.{:}}%
\@x{\@s{59.33} \.{\LET} p \.{\defeq} private [ n ] . preferences [ index ]
\.{\IN}}%
\@x{\@s{79.73} \.{\land} RunSimulation}%
\@x{\@s{79.73} \.{\land} p . status \.{=}\@w{not{-}processed}}%
\@x{\@s{79.73} \.{\land} InitializeJudgment ( p . package ,\, n ,\, p . level
)}%
\@x{\@s{79.73} \.{\land} private \.{'} \.{=} [ private {\EXCEPT}\@s{8.2}
{\bang} [ n ] \.{=} [ @ {\EXCEPT}}%
\@x{\@s{221.54} {\bang} . preferences \.{=} [ @ {\EXCEPT}}%
\@x{\@s{221.54} {\bang} [ index ] \.{=} [ @ {\EXCEPT}}%
\@x{\@s{221.54} {\bang} . status\@s{0.90} \.{=}\@w{started} ] ] ] ]}%
\@x{\@s{16.4}}%
\@y{\@s{37.5}%
\ensuremath{\.{\land} PrintT(\mbox{``}Started\.{\dots}\mbox{''})
}}%
\@xx{}%
\@x{\@s{82.0} \.{\land} {\UNCHANGED} {\langle} packages ,\, nextPackageId ,\,
isReproducible {\rangle}}%
\@x{\@s{32.8} \.{\lor}\@s{4.1} \E\, n \.{\in} Nodes \.{:}}%
\@x{\@s{48.01} \E\, j\@s{1.87} \.{\in} Range ( public . judgments ) \.{:}}%
\@x{\@s{59.33} \.{\LET} secretVote \.{\defeq} IsReproducible ( j . package )
\.{\IN}}%
\@x{\@s{59.33} \.{\land} RunSimulation}%
\@x{\@s{59.33}}%
\@y{\@s{0}%
    Nodes only build/judge if they have to
}%
\@xx{}%
\@x{\@s{59.33} \.{\land}\@s{4.1} \.{\lor}\@s{4.1} FutureCost ( n ) \.{>}
public . nodes [ n ] . wallet \.{-} 1}%
\@x{\@s{74.54} \.{\lor}\@s{4.10} j . owner \.{=} n}%
\@x{\@s{74.54} \.{\lor}\@s{4.10} j . package \.{\in} \{ p . package \.{:} p
\.{\in} Range ( private [ n ] . preferences ) \}}%
\@x{\@s{59.33} \.{\land}\@s{4.09} AddSecretJudgment ( j . id ,\, n ,\,
secretVote )}%
\@x{\@s{59.33} \.{\land}\@s{4.09} {\UNCHANGED} {\langle} private ,\, packages
,\, nextPackageId ,\, isReproducible {\rangle}}%
\@x{\@s{32.8} \.{\lor}\@s{4.09} \E\, n \.{\in} Nodes \.{:}}%
\@x{\@s{48.01} \E\, j\@s{1.87} \.{\in} Range ( public . judgments ) \.{:}}%
\@x{\@s{59.33} \.{\land} RunSimulation}%
\@x{\@s{59.33}}%
\@y{\@s{0}%
We assume that everyone \ensuremath{whoe \.{*}}needs* to vote will have the
time to do so
}%
\@xx{}%
\@x{\@s{59.33} \.{\land}\@s{4.1} \A\, nonVoter \.{\in} Nodes
\.{\,\backslash\,} SecretJudges ( j ) \.{:}}%
\@x{\@s{85.86} FutureCost ( nonVoter ) \.{\leq} public . nodes [ nonVoter ] .
wallet \.{-} 1}%
\@x{\@s{59.33} \.{\land}\@s{4.09} EndSecretSubmissions ( j . id ,\, n )}%
\@x{\@s{59.33} \.{\land} {\UNCHANGED} {\langle} private ,\, packages ,\,
nextPackageId ,\, isReproducible {\rangle}}%
\@x{\@s{32.8} \.{\lor}\@s{4.09} \E\, n \.{\in} Nodes \.{:}}%
\@x{\@s{48.01} \E\, j\@s{1.87} \.{\in} Range ( public . judgments ) \.{:}}%
\@x{\@s{59.33} \.{\LET} openVote \.{\defeq} IsReproducible ( j . package )
\.{\IN}}%
\@x{\@s{59.33} \.{\land} RunSimulation}%
\@x{\@s{59.33} \.{\land}\@s{4.1} \.{\lor}\@s{4.1} FutureCost ( n ) \.{>}
public . nodes [ n ] . wallet \.{-} 1}%
\@x{\@s{74.54} \.{\lor}\@s{4.10} j . owner \.{=} n}%
\@x{\@s{74.54} \.{\lor}\@s{4.10} j . package \.{\in} \{ p . package \.{:} p
\.{\in} Range ( private [ n ] . preferences ) \}}%
\@x{\@s{59.33} \.{\land} ShowJudgment ( j . id ,\, n ,\, openVote )}%
\@x{\@s{59.33} \.{\land} {\UNCHANGED} {\langle} private ,\, packages ,\,
nextPackageId ,\, isReproducible {\rangle}}%
\@x{\@s{32.8} \.{\lor}\@s{4.09} \E\, n \.{\in} Nodes \.{:}}%
\@x{\@s{48.01} \E\, j\@s{1.87} \.{\in} Range ( public . judgments ) \.{:}}%
\@x{\@s{59.33} \.{\land} RunSimulation}%
\@x{\@s{59.33}}%
\@y{\@s{0}%
We assume that everyone \ensuremath{whoe \.{*}}needs* to vote will have the
time to do so
}%
\@xx{}%
\@x{\@s{59.33} \.{\land}\@s{4.1} \A\, nonVoter \.{\in} Nodes
\.{\,\backslash\,} OpenJudges ( j ) \.{:}}%
\@x{\@s{85.86} FutureCost ( nonVoter ) \.{\leq} public . nodes [ nonVoter ] .
wallet \.{-} 1}%
\@x{\@s{59.33} \.{\land}\@s{4.09} CloseJudgment ( j . id ,\, n )}%
\@x{\@s{59.33} \.{\land}\@s{4.09} \E\, preferenceIndex \.{\in} {\DOMAIN}
private [ n ] . preferences \.{:}}%
\@x{\@s{85.86} \.{\land}\@s{4.1} private [ n ] . preferences [
        preferenceIndex ] . package \.{=} j . package}%
\@x{\@s{85.86} \.{\land}\@s{4.09} private \.{'} \.{=} [ private {\EXCEPT}
    {\bang} [ n ] . preferences [ preferenceIndex ] . status \.{=}\@w{processed}
]}%
\@x{\@s{59.33} \.{\land} {\UNCHANGED}\@s{4.1} {\langle} packages ,\,
nextPackageId ,\, isReproducible {\rangle}}%
\@x{\@s{32.8} \.{\lor}\@s{4.09} \E\, n \.{\in} Nodes \.{:}}%
\@x{\@s{48.01} \.{\land} RunSimulation}%
\@x{\@s{48.01} \.{\land} AddPreferencedPackage ( n )}%
\@x{\@s{48.01} \.{\land} {\UNCHANGED} {\langle} public ,\, packages ,\,
isReproducible {\rangle}}%
\@x{\@s{32.8} \.{\lor}\@s{4.09} {\UNCHANGED} {\langle} private ,\, packages
,\, public ,\, nextPackageId ,\, isReproducible {\rangle}}%
\@pvspace{16.0pt}%
\@x{ Spec \.{\defeq} Init \.{\land} {\Box} [ Next ]_{ {\langle} public ,\,
private ,\, packages ,\, nextPackageId ,\, isReproducible {\rangle}}
\.{\land} Fairness}%
\@pvspace{16.0pt}%
\@x{}\bottombar\@xx{}%
\end{document}

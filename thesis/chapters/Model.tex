\chapter{TLA\textsuperscript+ model}

To develop and study the judgment algorithm, a model was created as a TLA\textsuperscript+ specification. This chapter seeks to describe how this model is structured and to detail its core components. The specification has been split into three parts with some details removed to make the explanation more manageable for the reader. A full version of the specification can be found in appendix \ref{ch:TLAPlusFullModel}. This section has been divided into \ref{subsec:ModelData} on the data structures used in the model, chaincode definitions in \ref{subsec:ModelChaincode} and how system actors interact under \ref{subsec:ModelSpec}.

\lstset{
    escapechar=µ
    , numbers=left
    , firstnumber=auto}

\section{Data structures}
\label{subsec:ModelData}

The models main data structures, variables and module setup are shown in listing \ref{lst:ModelData}. It begins by declaring the module name and then listing its module dependencies on line \ref{line:ExtendedModules}. These modules makes available operators for working with real numbers and list-like sequences, as well as ensuring that sets are finite in the model. The last point is mainly important for the model checker. Lines \ref{line:VariableDecl}-\ref{line:ConstantDecl} declares all the models mutable variables as well as the model checking goal as an immutable constant. While this constant is declared in the module, it is set by the configuration of the model checker and can be thought of as a parameter of the model. The three variables respectively represent: public information known to all system actors, private information only known by a single actor and a counter to generate unique package names.

Lines \ref{line:AllNodes}-\ref{line:MaliciousNodes} define sets containing all the system actors, called \textit{nodes} in the model, and which system actors that act maliciously. The last lines define the initial state of the \texttt{public} and \texttt{private} variables and the correct information on whether a package is reproducible or not. As can be noted, the \texttt{private} variable contains a mapping from system actor to their preferred packages. These preferred packages are disjunct between nodes to simplify the rest of the model. Each package preference stores how many nodes that has to judge the package for the node to be satisfied with the result and whether the package has been judged yet. While the level is set at a hard coded value initially, this is also only to simplify the model. The \texttt{public} variable has instead two different parts. One describes the number of tokens a node has and the other stores the package judgments created during the execution of model checking. Because only one node starts of with tokens, they are the only ones who can initialize a package judgment at the start.

\begin{lstlisting}[	  caption=Variables and TLA\textsuperscript+ model dependencies
					, label=lst:ModelData]
	µ
	\begin{tlatex}
\@x{}\moduleLeftDash\@xx{ {\MODULE} Judgment}\moduleRightDash\@xx{}%
\@x{ {\EXTENDS} TLC ,\, FiniteSets ,\, Reals ,\, Sequences}%
\label{line:ExtendedModules}
\@pvspace{8.0pt}%
\@x{ {\VARIABLE} public ,\, private ,\, nextPackageId}%
\label{line:VariableDecl}
\@pvspace{8.0pt}%
\@x{ {\CONSTANT} NumberOfClosedJudgmentsGoal}%
\label{line:ConstantDecl}
\@pvspace{16.0pt}%
\@x{ Nodes \.{\defeq} \{\@w{node1} ,\,\@w{node2} ,\,\@w{node3} ,\,\@w{node4}
\}}%
\label{line:AllNodes}
\@x{ MaliciousNodes \.{\defeq} \{\@w{node4} \}}%
\label{line:MaliciousNodes}
\@pvspace{8.0pt}%
\@x{ InitialPrivate\@s{9.53} \.{\defeq} private \.{=} [}%
\@x{\@s{16.4} node1 \.{\mapsto} [}%
\@x{\@s{32.8} preferences \.{\mapsto} {\langle}}%
\@x{\@s{49.19} [ package \.{\mapsto}\@w{package1} ,\, level \.{\mapsto} 2 ,\,
status \.{\mapsto}\@w{not{-}processed} ] ,\,}%
\@x{\@s{49.19} [ package \.{\mapsto}\@w{package2} ,\, level \.{\mapsto} 2 ,\,
status \.{\mapsto}\@w{not{-}processed} ]}%
\@x{\@s{32.8} {\rangle}}%
\@x{\@s{16.4} ] ,\,}%
\@x{\@s{16.4} node2 \.{\mapsto} [}%
\@x{\@s{32.8} preferences \.{\mapsto} {\langle}}%
\@x{\@s{49.19} [ package \.{\mapsto}\@w{package3} ,\, level \.{\mapsto} 2 ,\,
status \.{\mapsto}\@w{not{-}processed} ] ,\,}%
\@x{\@s{49.19} [ package \.{\mapsto}\@w{package4} ,\, level \.{\mapsto} 2 ,\,
status \.{\mapsto}\@w{not{-}processed} ]}%
\@x{\@s{32.8} {\rangle}}%
\@x{\@s{16.4} ] ,\,}%
\@x{\@s{16.4} node3 \.{\mapsto} [}%
\@x{\@s{32.8} preferences \.{\mapsto} {\langle}}%
\@x{\@s{49.19} [ package \.{\mapsto}\@w{package5} ,\, level \.{\mapsto} 2 ,\,
status \.{\mapsto}\@w{not{-}processed} ] ,\,}%
\@x{\@s{49.19} [ package \.{\mapsto}\@w{package6} ,\, level \.{\mapsto} 2 ,\,
status \.{\mapsto}\@w{not{-}processed} ]}%
\@pvspace{8.0pt}%
\@x{\@s{32.8} {\rangle}}%
\@x{\@s{16.4} ] ,\,}%
\@x{\@s{16.4} node4 \.{\mapsto} [}%
\@x{\@s{32.8} preferences \.{\mapsto} {\langle}}%
\@x{\@s{49.19} [ package \.{\mapsto}\@w{package7} ,\, level \.{\mapsto} 2 ,\,
status \.{\mapsto}\@w{not{-}processed} ] ,\,}%
\@x{\@s{49.19} [ package \.{\mapsto}\@w{package8} ,\, level \.{\mapsto} 2 ,\,
status \.{\mapsto}\@w{not{-}processed} ]}%
\@pvspace{8.0pt}%
\@x{\@s{32.8} {\rangle}}%
\@x{\@s{16.4} ]}%
\@x{ ]}%
\@pvspace{8.0pt}%
\@x{ InitialPublic \.{\defeq} public \.{=} [}%
\@x{\@s{16.4} nodes \.{\mapsto} [}%
\@x{\@s{32.8} node1 \.{\mapsto} [ wallet \.{\mapsto} 3 ] ,\,}%
\@x{\@s{32.8} node2 \.{\mapsto} [ wallet \.{\mapsto} 0 ] ,\,}%
\@x{\@s{32.8} node3 \.{\mapsto} [ wallet \.{\mapsto} 0 ] ,\,}%
\@x{\@s{32.8} node4 \.{\mapsto} [ wallet \.{\mapsto} 0 ]}%
\@x{\@s{16.4} ] ,\,}%
\@x{\@s{16.4} judgments \.{\mapsto} {\langle} {\rangle}}%
\label{line:EmptyJudgments}
\@x{ ]}%
\@pvspace{8.0pt}%
\@x{ IsReproducibleData \.{\defeq}\@s{8.2} [ package1 \.{\mapsto} {\TRUE}
,\,}%
\@x{\@s{114.37} package2 \.{\mapsto} {\FALSE} ,\,}%
\@x{\@s{114.37} package3 \.{\mapsto} {\TRUE} ,\,}%
\@x{\@s{114.37} package4 \.{\mapsto} {\TRUE} ,\,}%
\@x{\@s{114.37} package5 \.{\mapsto} {\TRUE} ,\,}%
\@x{\@s{114.37} package6 \.{\mapsto} {\TRUE} ,\,}%
\@x{\@s{114.37} package7 \.{\mapsto} {\TRUE} ,\,}%
\@x{\@s{114.37} package8 \.{\mapsto} {\TRUE} ,\,}%
\@x{\@s{114.37} package9 \.{\mapsto} {\FALSE} ,\,}%
\@x{\@s{114.37} package10 \.{\mapsto} {\FALSE} ,\,}%
\@x{\@s{114.37} package11 \.{\mapsto} {\FALSE} ,\,}%
\@x{\@s{114.37} package12 \.{\mapsto} {\FALSE} ,\,}%
\@x{\@s{114.37} package13 \.{\mapsto} {\FALSE} ,\,}%
\@x{\@s{114.37} package14 \.{\mapsto} {\FALSE} ,\,}%
\@x{\@s{114.37} package15 \.{\mapsto} {\FALSE} ,\,}%
\@x{\@s{114.37} package16 \.{\mapsto} {\FALSE} ,\,}%
\@x{\@s{114.37} package17 \.{\mapsto} {\FALSE} ,\,}%
\@x{\@s{114.37} package18 \.{\mapsto} {\FALSE} ,\,}%
\@x{\@s{114.37} package19 \.{\mapsto} {\FALSE} ,\,}%
\@x{\@s{114.37} package20 \.{\mapsto} {\FALSE} ]}%
	\end{tlatex}
	µ
\end{lstlisting}

\section{Chaincode}
\label{subsec:ModelChaincode}

Listings \ref{lst:ModelInitializeChaincode}, \ref{lst:ModelSubmitHiddenChaincode}, \ref{lst:ModelEndHiddenChaincode}, \ref{lst:ModelRevealVoteChaincode} and \ref{lst:ModelEndJudgmentChaincode} contains the TLA\textsuperscript+ specifications of what would be written as chaincode in a system implementation. They are defined as a number of operators, each one representing a distinct chaincode. Each operator take some parameters such as who is invoking the chaincode, which judgment or package it relates to and other relevant information. They then verify whether the particular transaction is allowed or not and, if it was, updates the \texttt{public} variable. This is exemplified in the operator for initializing a new package judgment \texttt{InitializeJudgment} on lines \ref{line:InitializeJudgmentBegin}-\ref{line:InitializeJudgmentEnd}. Firstly, checks are made to ensure that the invoker \texttt{owner} has enough funds to initialize a package judgment and that there are no other judgments initialized by them currently active. If these predicates are fulfilled, a corresponding judgment is appended to the state. Note that all of these operators are in practice written as logical conjunctions, where every sub-expression must be true for the entire operator to be true. While intuitive for predicates, this is not quite obvious with an action such as on line \ref{line:InitializeJudgmentEnd}. For model checking and software development purposes, this can be thought of as an assignment where the \texttt{public} variable is updated. From a logical point of view, though, this is an equivalence check which is true only for those behaviors where the \texttt{public} variable in the next state is exactly the one as the current with the exception of an appended judgment to the \texttt{public.judgments} sequence. In practice these become equivalent statements because any behavior where the relationship between \texttt{public'} and \texttt{public} is different (and the equivalence on line \ref{line:InitializeJudgmentEnd} therefore is false) are irrelevant as they do not follow the specification.

To clarify the notation, \begin{tlatex}{\LET \IN}\end{tlatex} allows locally scoped definitions and in the \begin{tlatex}{\EXCEPT}\end{tlatex} at line \ref{line:InitializeJudgmentEnd}, \begin{tlatex}{\bang}\end{tlatex} is short for \texttt{public} and @ is short for \texttt{public.judgments}. Definitions for operators \texttt{Cost} and \texttt{ActiveJudgments} can be found in appendix \ref{ch:TLAPlusFullModel}.

\begin{lstlisting}[caption=Chaincode for initializing a package judgment, label=lst:ModelInitializeChaincode]
	µ
	\begin{tlatex}
\@x{ InitializeJudgment ( package ,\, owner ,\, targetVotes ) \.{\defeq}}%
\label{line:InitializeJudgmentBegin}
\@x{\@s{16.4}}%
\@y{\@s{0}%
    Guards
}%
\@xx{}%
\@x{\@s{16.4} \.{\land} Cost ( targetVotes ) \.{\leq} public . nodes [ owner
    ] . wallet}%
\@x{\@s{16.4} \.{\land} \A\, j \.{\in} ActiveJudgments \.{:} j . owner
\.{\neq} owner}%
\@x{\@s{16.4}}%
\@y{\@s{0}%
    Update
}%
\@xx{}%
\@x{\@s{16.4}}%
\@y{\@s{0}%
    \ensuremath{TODO}: Better ids. Currently there can only be one judgment per
    package.
}%
\@xx{}%
\@x{\@s{16.4} \.{\land} \.{\LET} judgmentId \.{\defeq} package}%
\@x{\@s{47.91} judgment \.{\defeq} [ id \.{\mapsto} judgmentId ,\,}%
\@x{\@s{110.39} tally \.{\mapsto} [ for \.{\mapsto} 0 ,\, against \.{\mapsto}
0 ] ,\,}%
\@x{\@s{110.39} finalResult \.{\mapsto}\@w{undecided} ,\,}%
\@x{\@s{110.39} package \.{\mapsto} package ,\,}%
\@x{\@s{110.39} owner \.{\mapsto} owner ,\,}%
\@x{\@s{110.39} targetVotes \.{\mapsto} targetVotes ,\,}%
\@x{\@s{110.39} status \.{\mapsto}\@w{active} ,\,}%
\@x{\@s{110.39} phase \.{\mapsto}\@w{secretVotes} ,\,}%
\@x{\@s{110.39} secretJudgments \.{\mapsto} {\langle} {\rangle} ,\,}%
\@x{\@s{110.39} openJudgments \.{\mapsto} {\langle} {\rangle} ] \.{\IN}}%
\@x{\@s{31.61} public \.{'} \.{=} [ public {\EXCEPT} {\bang} . judgments
\.{=} Append ( @ ,\, judgment ) ]}%
\label{line:InitializeJudgmentEnd}
	\end{tlatex}
	µ
\end{lstlisting}

\subsection{Submitting a hidden vote}
\label{subsec:ModelSubmitHiddenChaincode}

To add a hidden vote to a package judgment, the operator in listing \ref{lst:ModelSubmitHiddenChaincode} is used. It takes the id of the judgment, the name of the node adding the vote, and the vote in question as parameters. The disjunctions on lines \ref{line:IfEmptyExampleBegin}-\ref{line:IfEmptyExampleEnd} is a pattern common for the rest of the chaincode specifications. If there are no judgments then line \ref{line:IfEmptyExampleBegin} ensures that this operator does nothing, otherwise some \texttt{index} is picked corresponding to the position of a particular judgment in the \texttt{public.judgments} sequence. This works because sequences in TLA\textsuperscript+ are essentially functions from positions to values and so the domain of a sequence is the set of all those positions.

The rest of this operator is fairly straightforward, but line \ref{line:RangeExample} could need an explanation. Symmetric to the domain of the sequence, its range is a set of its values. Line \ref{line:RangeExample} can then be understood as testing if the set of secret votes submitted by node \texttt{judge} is empty.

\begin{lstlisting}[caption=Chaincode for adding a hidden vote, label=lst:ModelSubmitHiddenChaincode]
	µ
	\begin{tlatex}
        \@x{ AddSecretJudgment ( judgmentId ,\, judge ,\, secretVote ) \.{\defeq}}%
        \@x{\@s{16.4} \.{\lor}\@s{4.1} Len ( public . judgments ) \.{=} 0}%
        \label{line:IfEmptyExampleBegin}
        \@x{\@s{16.4} \.{\lor}\@s{4.09} \E\, index \.{\in} {\DOMAIN} public .
        judgments \.{:}}%
        \label{line:IfEmptyExampleEnd}
        \@x{\@s{42.93} \.{\LET} j \.{\defeq} public . judgments [ index ] \.{\IN}}%
        \@x{\@s{63.33} \.{\land} j . id \.{=} judgmentId}%
        \@x{\@s{63.33} \.{\land} j . status \.{=}\@w{active}}%
        \@x{\@s{63.33} \.{\land} j . phase \.{=}\@w{secretVotes}}%
        \@x{\@s{63.33}}%
        \@y{\@s{0}%
        Has \mbox{'}judge\mbox{'} added their vote before\.{?}
        }%
        \@xx{}%
        \@x{\@s{63.33} \.{\land} \{ sj \.{\in} Range ( j . secretJudgments ) \.{:} sj
        . judge \.{=} judge \} \.{=} \{ \}}%
        \label{line:RangeExample}
        \@x{\@s{63.33}}%
        \@y{\@s{0}%
            Update
        }%
        \@xx{}%
        \@x{\@s{63.33} \.{\land}\@s{4.1} public \.{'} \.{=}}%
        \@x{\@s{94.94} [ public {\EXCEPT} {\bang} . judgments [ index ] .
        secretJudgments \.{=}}%
        \@x{\@s{110.02} Append ( @ ,\, [ judge \.{\mapsto} judge ,\,}%
        \@x{\@s{164.08} vote \.{\mapsto} secretVote ] ) ]}%
    \end{tlatex}
    µ
\end{lstlisting}

\subsection{End hidden vote submissions}
\label{subsec:ModelEndHiddenChaincode}

After enough hidden votes have been added to a package judgment, this phase of the process can be ended by the node that initialized it. The requirements for this are listed on lines \ref{line:EndHiddenGuardBegin}-\ref{line:EndHiddenGuardEnd} in listing \ref{lst:ModelEndHiddenChaincode}. Most notably, \emph{enough} hidden votes is defined to guarantee that there are at least \texttt{targetVotes} number of votes on one side for or against the package being reproducible. In other words, it is guaranteed that at least one side has at least \texttt{targetVotes} number of votes.

\begin{lstlisting}[caption=Chaincode stopping submissions of hidden votes, label=lst:ModelEndHiddenChaincode]
	µ
	\begin{tlatex}
        \@x{ EndSecretSubmissions ( judgmentId ,\, owner ) \.{\defeq}}%
        \@x{\@s{16.4} \.{\lor} Len ( public . judgments ) \.{=} 0}%
        \@x{\@s{16.4} \.{\lor} \E\, index \.{\in} {\DOMAIN} public . judgments \.{:}}%\@x{\@s{31.61} \.{\LET} j \.{\defeq} public . judgments [ index ] \.{\IN}}%
        \@x{\@s{52.01} \.{\land} j . id \.{=} judgmentId}%
        \label{line:EndHiddenGuardBegin}
        \@x{\@s{52.01} \.{\land} j . status \.{=}\@w{active}}%
        \@x{\@s{52.01} \.{\land} j . phase\@s{2.49} \.{=}\@w{secretVotes}}%
        \@x{\@s{52.01} \.{\land} j . owner \.{=} owner}%
        \@x{\@s{52.01} \.{\land} 2 \.{*} j . targetVotes \.{-} 1 \.{\leq} Len ( j .
        secretJudgments )}%
        \label{line:EndHiddenGuardEnd}
        \@x{\@s{52.01} \.{\land} public \.{'} \.{=} [ public {\EXCEPT} {\bang} .
        judgments [ index ] . phase}%
        \@x{\@s{116.10} \.{=}\@w{openVotes} ]}%
    \end{tlatex}
    µ
\end{lstlisting}

\subsection{Revealing a vote}
\label{subsec:ModelRevealVoteChaincode}

To reveal ones hidden vote, the operator \texttt{ShowJudgment} shown in listing \ref{lst:ModelRevealVoteChaincode} is invoked. If the invocation is valid, \ie there is a hidden vote by this node which has not been revealed, the vote count for or against the reproducibility of the package is updated. The vote itself is also appended to a sequence of all revealed votes.

\begin{lstlisting}[caption=Chaincode for revealing a vote, label=lst:ModelRevealVoteChaincode]
	µ
	\begin{tlatex}
        \@x{ ShowJudgment ( judgmentId ,\, judge ,\, openVote ) \.{\defeq}}%
        \@x{\@s{16.4} \.{\lor} Len ( public . judgments ) \.{=} 0}%
        \@x{\@s{16.4} \.{\lor} \E\, index \.{\in} {\DOMAIN} public . judgments \.{:}}%
        \@x{\@s{31.61} \.{\LET} j \.{\defeq} public . judgments [ index ] \.{\IN}}%
        \@x{\@s{52.01} \.{\land} j . id \.{=} judgmentId}%
        \@x{\@s{52.01} \.{\land} j . status \.{=}\@w{active}}%
        \@x{\@s{52.01} \.{\land} j . phase \.{=}\@w{openVotes}}%
        \@x{\@s{52.01}}%
        \@y{\@s{0}%
        Has \mbox{'}judge\mbox{'} added their vote before\.{?}
        }%
        \@xx{}%
        \@x{\@s{52.01} \.{\land} \{ sj \.{\in} Range ( j . secretJudgments ) \.{:} sj
        . judge \.{=} judge \} \.{\neq} \{ \}}%
        \@x{\@s{52.01}}%
        \@y{\@s{0}%
        Has \mbox{'}judge\mbox{'} showed their vote before\.{?}
        }%
        \@xx{}%
        \@x{\@s{52.01} \.{\land} \{ oj \.{\in} Range ( j . openJudgments ) \.{:} oj .
        judge \.{=} judge \} \.{=} \{ \}}%
        \@x{\@s{52.01}}%
        \@y{\@s{0}%
            Secret and open votes should be the same
        }%
        \@xx{}%
        \@x{\@s{52.01}}%
        \@y{\@s{0}%
            This validation is done with \ensuremath{HMAC} in practice
        }%
        \@xx{}%
        \@x{\@s{52.01} \.{\land} \E\, sj \.{\in} Range ( j . secretJudgments ) \.{:}}%
        \@x{\@s{67.22} \.{\land}\@s{4.1} sj . judge\@s{5.50} \.{=} judge}%
        \@x{\@s{67.22} \.{\land}\@s{4.09} openVote \.{=} sj . vote}%
        \@x{\@s{52.01}}%
        \@y{\@s{0}%
            Update
        }%
        \@xx{}%
        \@x{\@s{52.01} \.{\land}\@s{4.1} public \.{'} \.{=} [ public {\EXCEPT}
            {\bang} . judgments [ index ] \.{=} [ @ {\EXCEPT}}%
        \@x{\@s{116.10} {\bang} . openJudgments \.{=} Append ( @ ,\,}%
        \@x{\@s{131.65} [\@s{8.2} judge \.{\mapsto} judge ,\,}%
        \@x{\@s{142.62} vote \.{\mapsto} openVote ] ) ,\,}%
        \@x{\@s{116.10} {\bang} . tally . for \.{=} {\IF} openVote \.{=}\@w{TRUE}}%
        \@x{\@s{181.02} \.{\THEN} @ \.{+} 1}%
        \@x{\@s{181.02} \.{\ELSE} @ ,\,}%
        \@x{\@s{116.10} {\bang} . tally . against \.{=} {\IF} openVote
        \.{=}\@w{FALSE}}%
        \@x{\@s{177.89} \.{\THEN} @ \.{+} 1}%
        \@x{\@s{177.89} \.{\ELSE} @ ] ]}%
    \end{tlatex}
    µ
\end{lstlisting}

\subsection{End package judgment}
\label{subsec:ModelEndJudgmentChaincode}

The final step of a package judgment is begun by the node who initialized it invoking the \texttt{CloseJudgment} operator shown in listing \ref{lst:ModelEndJudgmentChaincode}. Similar to section \ref{subsec:ModelEndHiddenChaincode}, a package judgment can only be ended when the number of revealed votes is enough to ensure that there are at least \texttt{targetVotes} number of votes on one side for or against reproducibility. If their is a majority for one side, the \texttt{Rewards} operator seen in listing \ref{lst:ModelRewardChaincode} updates the nodes' wallets, otherwise the package judgment is closed and marked with \emph{undecided} as a final result.

\begin{lstlisting}[caption=Chaincode to end a package judgment, label=lst:ModelEndJudgmentChaincode]
	µ
	\begin{tlatex}
        \@x{ CloseJudgment ( judgmentId ,\, owner ) \.{\defeq}}%
        \@x{\@s{16.4} \.{\lor}\@s{4.1} Len ( public . judgments ) \.{=} 0}%
        \@x{\@s{16.4} \.{\lor}\@s{4.09} \E\, index \.{\in} {\DOMAIN} public .
        judgments \.{:}}%
        \@x{\@s{42.93} \.{\LET} j \.{\defeq} public . judgments [ index ]}%
        \@x{\@s{63.33} finalResult \.{\defeq} ToString ( j . tally . for \.{>} j .
        tally . against ) \.{\IN}}%
        \@x{\@s{63.33} \.{\land}\@s{4.1} j . id \.{=} judgmentId}%
        \@x{\@s{63.33} \.{\land}\@s{4.09} j . status \.{=}\@w{active}}%
        \@x{\@s{63.33} \.{\land}\@s{4.09} j . phase\@s{2.49} \.{=}\@w{openVotes}}%
        \@x{\@s{63.33} \.{\land}\@s{4.09} j . owner \.{=} owner}%
        \@x{\@s{63.33} \.{\land} 2 \.{*} j . targetVotes \.{-} 1 \.{\leq} Len ( j .
        openJudgments )}%
        \@x{\@s{78.54}}%
        \@y{\@s{0}%
            We have a majority!
        }%
        \@xx{}%
        \@x{\@s{63.33} \.{\land}\@s{4.09} \.{\lor}\@s{4.1} \.{\land}\@s{4.1} j .
        tally . for \.{\neq} j . tally . against}%
        \@x{\@s{93.75} \.{\land}\@s{4.09} public \.{'} \.{=} [ public {\EXCEPT}}%
        \@x{\@s{125.36} {\bang} . judgments [ index ] \.{=}}%
        \@x{\@s{140.91} [ @ {\EXCEPT}}%
        \@x{\@s{155.56} {\bang} . status \.{=}\@w{closed} ,\,}%
        \@x{\@s{155.56} {\bang} . phase \.{=}\@w{judgment} ,\,}%
        \@x{\@s{155.56} {\bang} . finalResult \.{=} finalResult ] ,\,}%
        \@x{\@s{125.36} {\bang} . nodes \.{=} Rewards ( j ) ]}%
        \@x{\@s{78.54}}%
        \@y{\@s{0}%
            Everyone has voted, and we don\mbox{'}t have a majority
        }%
        \@xx{}%
        \@x{\@s{78.54} \.{\lor}\@s{4.1} \.{\land}\@s{4.1} j . tally . for \.{=} j .
        tally . against}%
        \@x{\@s{93.75} \.{\land}\@s{4.09} public \.{'} \.{=} [ public {\EXCEPT}
            {\bang} . judgments [ index ] \.{=} [ @ {\EXCEPT}}%
        \@x{\@s{157.84} {\bang} . status \.{=}\@w{closed} ,\,}%
        \@x{\@s{157.84} {\bang} . finalResult \.{=}\@w{undecided} ] ]}%
    \end{tlatex}
    µ
\end{lstlisting}

The \texttt{Rewards} operator and its two sub-operators \texttt{WalletUpdatesForVoters} and \texttt{WalletUpdatesForNotVoting} in listing \ref{lst:ModelRewardChaincode} work by updating the \texttt{public.nodes} wallets. Together, they partition the nodes in four different cases for how these updates are done based on their respective role in the judgment: the initializer of the package judgment pays out a cost related to \texttt{targetVotes}, minority voters' wallets neither increase nor decrease, majority voters receive a reward based on how quickly they added their vote and a bonus token for the correctness of their vote, and the rest loose one token for not voting.

\begin{lstlisting}[caption=Operator updating wallets for judgment, label=lst:ModelRewardChaincode]
	µ
	\begin{tlatex}
        \@x{ {\RECURSIVE} WalletUpdatesForVoters ( \_ ,\, \_ ,\, \_ ,\, \_ )}%
        \@x{ WalletUpdatesForVoters ( judgment ,\, votes ,\, reward ,\, updates )
        \.{\defeq}}%
        \@x{\@s{16.4} \.{\LET} finalResult \.{\defeq} ToString ( judgment . tally .
        for \.{>} judgment . tally . against )}%
        \@x{\@s{36.79} judge \.{\defeq} Head ( votes ) . judge}%
        \@x{\@s{36.79} vote \.{\defeq} Head ( votes ) . vote}%
        \@x{\@s{16.4} \.{\IN}}%
        \@x{\@s{32.8} {\IF} Len ( votes ) \.{=} 0}%
        \@x{\@s{49.05} \.{\THEN} updates}%
        \@x{\@s{32.8} \.{\ELSE} {\IF} judge \.{=} judgment . owner}%
        \@x{\@s{49.19}}%
        \@y{\@s{0}%
        Skip the owner
        }%
        \@xx{}%
        \@x{\@s{49.19} \.{\THEN} WalletUpdatesForVoters (}%
        \@x{\@s{65.6} judgment ,\, Tail ( votes ) ,\, reward ,\, updates )}%
        \@x{\@s{32.8} \.{\ELSE} {\IF} vote \.{\neq} finalResult \.{\THEN} \.{\LET}}%
        \@x{\@s{49.19}}%
        \@y{\@s{0}%
        Minority voter
        }%
        \@xx{}%
        \@x{\@s{65.6} newUpdates \.{\defeq} [ updates {\EXCEPT} {\bang} [ judge ] .
        wallet \.{=} @ ]}%
        \@x{\@s{49.19} \.{\IN} WalletUpdatesForVoters (}%
        \@x{\@s{73.69} judgment ,\, Tail ( votes ) ,\, reward ,\, newUpdates )}%
        \@x{\@s{32.8} \.{\ELSE} \.{\LET}}%
        \@x{\@s{49.19}}%
        \@y{\@s{0}%
        Majority voter
        }%
        \@xx{}%
        \@x{\@s{65.6} newUpdates \.{\defeq} [ updates {\EXCEPT} {\bang} [ judge ] .
        wallet \.{=} @ \.{+} reward \.{+} 1 ]}%
        \@x{\@s{73.69} newReward \.{\defeq} {\IF} reward \.{=} 0 \.{\THEN} 0
        \.{\ELSE} reward \.{-} 1}%
        \@x{\@s{53.29} \.{\IN} WalletUpdatesForVoters (}%
        \@x{\@s{90.09} judgment ,\, Tail ( votes ) ,\, newReward ,\, newUpdates )}%
        \@pvspace{8.0pt}%

        \@x{ {\RECURSIVE} WalletUpdatesForNotVoting ( \_ ,\, \_ )}%
        \@x{ WalletUpdatesForNotVoting ( nonVoters ,\, updates ) \.{\defeq}}%
        \@x{\@s{16.4} {\IF} nonVoters \.{=} \{ \}}%
        \@x{\@s{32.65} \.{\THEN} updates}%
        \@x{\@s{32.65} \.{\ELSE}}%
        \@x{\@s{49.05} \.{\LET} v \.{\defeq} {\CHOOSE} voter \.{\in} nonVoters \.{:}
        {\TRUE}}%
        \@x{\@s{73.55} newUpdates \.{\defeq} [ updates {\EXCEPT} {\bang} [ v ] .
        wallet \.{=} @ \.{-} 1 ]}%
        \@x{\@s{49.05} \.{\IN}\@s{4.09} WalletUpdatesForNotVoting ( nonVoters
        \.{\,\backslash\,} \{ v \} ,\, newUpdates )}%

        \@x{ Rewards ( judgment ) \.{\defeq}}%
        \@x{\@s{16.4} \.{\LET} openVotes\@s{4.11} \.{\defeq} judgment .
        openJudgments}%
        \@x{\@s{36.79} topReward\@s{2.12} \.{\defeq} judgment . targetVotes}%
        \@x{\@s{36.79} wallets \.{\defeq} public . nodes}%
        \@x{\@s{36.79} openVoters \.{\defeq} OpenJudges ( judgment )}%
        \@x{\@s{36.79} nonVoters\@s{1.63} \.{\defeq} Nodes \.{\,\backslash\,}
        openVoters}%
        \@x{\@s{36.79} ownerCost \.{\defeq} Cost ( judgment . targetVotes )}%
        \@x{\@s{36.79} updatesForVoters \.{\defeq}}%
        \@x{\@s{53.19} WalletUpdatesForVoters ( judgment ,\, openVotes ,\, topReward
        ,\, wallets )}%
        \@x{\@s{40.89} updatesForNonVoters \.{\defeq}}%
        \@x{\@s{57.29} WalletUpdatesForNotVoting ( nonVoters ,\, updatesForVoters )}%
        \@x{\@s{16.4} \.{\IN}\@s{4.09} [ updatesForNonVoters {\EXCEPT} {\bang} [
        judgment . owner ] . wallet \.{=} @ \.{-} ownerCost ]}%
    \end{tlatex}
    µ
\end{lstlisting}

\section{Behavior of system actors}
\label{subsec:ModelSpec}

The last main parts of the model are the assumptions for how the nodes behave, certain system properties and how a successful run of the model checker is defined. Line \ref{line:ModelSpec} of listing \ref{lst:ModelSpec} shows the main operator of the specification, defining the initial variable values according to \texttt{Init} and all their allowed changes by \texttt{Next}. This last operator is interesting in that it follows a slightly different pattern than the other ones. Rather than being mainly a large conjunction, \texttt{Next} is instead a set of disjunctions. For the TLC model checker, this implies that only one, and exactly one, of these sub-expressions will be executed for each state or else there will be a deadlock if all of the sub-expressions are false. The limit of at most one executed expression comes for the shortcut behavior of TLC which will only evaluate the first of all truthful expressions in a disjunction.

\begin{lstlisting}[caption=System actor behavior in TLA\textsuperscript+ specification
    , label=lst:ModelSpec]
µ
\begin{tlatex}
\@x{ Init \.{\defeq}}%
\@x{\@s{16.4} \.{\land} InitialPublic}%
\@x{\@s{16.4} \.{\land} InitialPrivate}%
\@x{\@s{16.4} \.{\land} nextPackageId \.{=} 9}%
\@pvspace{8.0pt}%
\@x{ Next \.{\defeq}}%
\@pvspace{8.0pt}%
\@x{\@s{33.24}}%
\@y{\@s{0}%
Initializing a package judgment
}%
\@xx{}%
\@x{\@s{33.24} \.{\lor}\@s{4.1} \E\, n \.{\in} Nodes \.{:}}%
\@x{\@s{48.45} \E\, index \.{\in} {\DOMAIN} private [ n ] . preferences
\.{:}}%
\@x{\@s{59.77} \.{\LET} p \.{\defeq} private [ n ] . preferences [ index ]
\.{\IN}}%
\@x{\@s{80.17} \.{\land} RunSimulation}%
\@x{\@s{80.17} \.{\land} p . status \.{=}\@w{not{-}processed}}%
\@x{\@s{80.17} \.{\land} InitializeJudgment ( p . package ,\, n ,\, p . level
)}%
\@x{\@s{80.17} \.{\land} private \.{'} \.{=}}%
\@x{\@s{95.38} [ private {\EXCEPT} {\bang} [ n ] . preferences [ index ] .
status \.{=}\@w{started} ]}%
\@x{\@s{80.17} \.{\land} {\UNCHANGED} {\langle} nextPackageId {\rangle}}%
\@pvspace{8.0pt}%
\@x{\@s{33.24}}%
\@y{\@s{0}%
Adding a secret vote
}%
\@xx{}%
\@x{\@s{33.24} \.{\lor}\@s{4.1} \E\, n \.{\in} Nodes \.{:}}%
\@x{\@s{48.45} \E\, j\@s{1.87} \.{\in} Range ( public . judgments ) \.{:}}%
\@x{\@s{59.77} \.{\LET} secretVote \.{\defeq} {\IF} n \.{\in} MaliciousNodes}%
\@x{\@s{152.12} \.{\THEN} {\lnot} IsReproducible ( j . package )}%
\@x{\@s{152.12} \.{\ELSE} IsReproducible ( j . package )}%
\@x{\@s{59.77} \.{\IN}}%
\@x{\@s{76.17} \.{\land} RunSimulation}%
\@x{\@s{76.17}}%
\@y{\@s{0}%
Nodes only build/judge if they have to
}%
\@xx{}%
\@x{\@s{76.17} \.{\land}\@s{4.1} \.{\lor}\@s{4.1} FutureCost ( n ) \.{>}
public . nodes [ n ] . wallet \.{-} 1}%
\@x{\@s{91.38} \.{\lor}\@s{4.09} j . owner \.{=} n}%
\@x{\@s{91.38} \.{\lor}\@s{4.09} j . package \.{\in} PreferredPackages ( n )}%
\@x{\@s{76.17} \.{\land}\@s{4.09} AddSecretJudgment ( j . id ,\, n ,\,
ToString ( secretVote ) )}%
\@x{\@s{76.17} \.{\land}\@s{4.09} {\UNCHANGED} {\langle} private ,\,
nextPackageId {\rangle}}%
\@pvspace{8.0pt}%
\@x{\@s{33.24}}%
\@y{\@s{0}%
End submission of secret votes
}%
\@xx{}%
\@x{\@s{33.24} \.{\lor}\@s{4.1} \E\, n \.{\in} Nodes \.{:}}%
\@x{\@s{48.45} \E\, j\@s{1.87} \.{\in} Range ( public . judgments ) \.{:}}%
\@x{\@s{59.77} \.{\land} RunSimulation}%
\@x{\@s{59.77}}%
\@y{\@s{0}%
Assume that every node that *needs* to vote will have the time to do so
}%
\@xx{}%
\@x{\@s{59.77} \.{\land}\@s{4.1} \A\, nonVoter \.{\in} Nodes
\.{\,\backslash\,} SecretJudges ( j ) \.{:}}%
\@x{\@s{86.30} \.{\lor}\@s{4.1} IsFinanciallyIncentivizedToVote ( nonVoter )}%
\@x{\@s{86.30} \.{\lor}\@s{4.09} n \.{\in} MaliciousNodes}%
\@x{\@s{59.77} \.{\land}\@s{4.09} EndSecretSubmissions ( j . id ,\, n )}%
\@x{\@s{59.77} \.{\land} {\UNCHANGED} {\langle} private ,\, nextPackageId
{\rangle}}%
\@pvspace{8.0pt}%
\@x{\@s{33.24}}%
\@y{\@s{0}%
Add an open vote
}%
\@xx{}%
\@x{\@s{33.24} \.{\lor}\@s{4.1} \E\, n \.{\in} Nodes \.{:}}%
\@x{\@s{48.45} \E\, j\@s{1.87} \.{\in} Range ( public . judgments ) \.{:}}%
\@x{\@s{59.77} \.{\LET} openVote \.{\defeq} {\IF} n \.{\in} MaliciousNodes}%
\@x{\@s{155.70} \.{\THEN} {\lnot} IsReproducible ( j . package )}%
\@x{\@s{155.70} \.{\ELSE} IsReproducible ( j . package )}%
\@x{\@s{59.77} \.{\IN}}%
\@x{\@s{76.17} \.{\land} RunSimulation}%
\@x{\@s{76.17} \.{\land}\@s{4.1} \.{\lor}\@s{4.1} FutureCost ( n ) \.{>}
public . nodes [ n ] . wallet \.{-} 1}%
\@x{\@s{91.38} \.{\lor}\@s{4.09} j . owner \.{=} n}%
\@x{\@s{91.38} \.{\lor}\@s{4.09} j . package \.{\in} PreferredPackages ( n )}%
\@x{\@s{76.17} \.{\land} ShowJudgment ( j . id ,\, n ,\, ToString ( openVote
) )}%
\@x{\@s{76.17} \.{\land} {\UNCHANGED} {\langle} private ,\, nextPackageId
{\rangle}}%
\@pvspace{8.0pt}%
\@x{\@s{33.24}}%
\@y{\@s{0}%
Close judgment
}%
\@xx{}%
\@x{\@s{33.24} \.{\lor}\@s{4.1} \E\, n \.{\in} Nodes \.{:}}%
\@x{\@s{48.45} \E\, j\@s{1.87} \.{\in} Range ( public . judgments ) \.{:}}%
\@x{\@s{59.77} \.{\land} RunSimulation}%
\@x{\@s{59.77}}%
\@y{\@s{0}%
Assume that every node that *needs* to vote will have the time to do so
}%
\@xx{}%
\@x{\@s{59.77} \.{\land}\@s{4.1} \A\, nonVoter \.{\in} Nodes
\.{\,\backslash\,} OpenJudges ( j ) \.{:}}%
\@x{\@s{86.30} \.{\lor} IsFinanciallyIncentivizedToVote ( nonVoter )}%
\@x{\@s{86.30} \.{\lor} n \.{\in} MaliciousNodes}%
\@x{\@s{59.77} \.{\land}\@s{4.09} CloseJudgment ( j . id ,\, n )}%
\@x{\@s{59.77}}%
\@y{\@s{0}%
Update the nodes preferences
}%
\@xx{}%
\@x{\@s{59.77} \.{\land}\@s{4.1} \E\, preferenceIndex \.{\in} {\DOMAIN}
private [ n ] . preferences \.{:}}%
\@x{\@s{86.30} \.{\land} private [ n ] . preferences [ preferenceIndex ] .
package}%
\@x{\@s{117.92} \.{=} j . package}%
\@x{\@s{86.30} \.{\land} private \.{'} \.{=}}%
\@x{\@s{97.42} [ private {\EXCEPT} {\bang} [ n ] . preferences [
preferenceIndex ] . status}%
\@x{\@s{116.59} \.{=}\@w{processed} ]}%
\@x{\@s{59.77} \.{\land} {\UNCHANGED}\@s{4.1} {\langle} nextPackageId
{\rangle}}%
\@pvspace{8.0pt}%
\@x{\@s{33.24}}%
\@y{\@s{0}%
Add additional package to node
}%
\@xx{}%
\label{line:ModelAddPackageBegin}
\@x{\@s{33.24} \.{\lor}\@s{4.1} \E\, n \.{\in} Nodes \.{:}}%
\@x{\@s{48.45} \.{\land} RunSimulation}%
\@x{\@s{48.45} \.{\land} AddPreferencedPackage ( n )}%
\@x{\@s{48.45} \.{\land} {\UNCHANGED} {\langle} public {\rangle}}%
\label{line:ModelAddPackageEnd}
\@pvspace{8.0pt}%
\@x{\@s{33.24}}%
\@y{\@s{0}%
End simulation
}%
\@xx{}%
\label{line:ModelEndStateBegin}
\@x{\@s{33.24} \.{\lor}\@s{4.1} \.{\land} {\lnot} RunSimulation}%
\@x{\@s{48.45} \.{\land} {\UNCHANGED} {\langle} private ,\, public ,\,
nextPackageId {\rangle}}%
\label{line:ModelEndStateEnd}
\@pvspace{16.0pt}%
\@x{ Spec\@s{1.46} \.{\defeq} Init \.{\land} {\Box} [ Next ]_{ {\langle}
public ,\, private ,\, nextPackageId {\rangle}}}%
\label{line:ModelSpec}
\end{tlatex}
µ
\end{lstlisting}

\subsection{Model checking goal}

Deadlocks happen if the model checker finds a state at which no sub-expression in the \texttt{Next} operator evaluates to true. While this is convenient for finding bugs in the specification, it can also lead to false-negatives if the state space of the model is designed to be finite. In other words, it is problematic if the model checker ends up in a deadlock when it is done. To combat this situation, the predicate \texttt{RunSimulation} shown in listing \ref{lst:RunSimulation} is used in \texttt{Next} to ensure the existence of a valid next state even after the model checking goal has been reached. Lines \ref{line:ModelEndStateBegin}-\ref{line:ModelEndStateEnd} in listing \ref{lst:ModelSpec} are true exactly when the model checking goal has been reached. At that point, the other sub-expressions in \texttt{Next} are false because they are dependent on \texttt{RunSimulation} to be true. When the goal has been fulfilled, the TLC model checker finds the sub-expression at lines \ref{line:ModelEndStateBegin}-\ref{line:ModelEndStateEnd} to be the only one that is true. Because that sub-expression also does not change any variables value, TLC infers that no new states can be made and the behavior finishes successfully.

\begin{lstlisting}[caption=Test to check if model checking goal is met
                    , label=lst:RunSimulation]
    µ
    \begin{tlatex}
        \@x{ RunSimulation \.{\defeq}}%
        \@x{\@s{16.4} Cardinality ( ClosedJudgments ) \.{<}
        NumberOfClosedJudgmentsGoal}%
    \end{tlatex}
    µ
\end{lstlisting}

\subsection{Adding packages}

In the initial state, each node has a number of preferred packages as a means of goals. As stated in section \ref{subsec:ModelData}, these packages are set statically to simplify the model. To introduce a bit of dynamic behavior into the model, the sub-expression at lines \ref{line:ModelAddPackageBegin}-\ref{line:ModelAddPackageEnd} in listing \ref{lst:ModelSpec} has the possibility of adding a package to one of the nodes. The target vote level of that package is based on the wallet and any other current preferred packages of a node. This simulates new packages and package versions being added to the eco-system and nodes being willing to pay more tokens to get a more certain result for packages if they can. 



\subsection{System actor assumptions}


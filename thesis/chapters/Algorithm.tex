\chapter{Judgment algorithm}

\section{Algorithm: hidden vote judgment}

A description of the voting scheme follows and is split into a single-package and a multi-package case. The single-package algorithm denote how a judgment is made on the reproducibility of a single package, while the multi-package variant generalizes this algorithm to scenarios with multiple packages through a reward scheme to motivate builders.

\subsection{Single-package algorithm}
\label{subsec:AlgorithmSingle}

A definition of \textit{what} the single-package judgment algorithm does can be stated as:

\begin{quote}
	Given the name of a \texttt{.buildinfo} file $n_p$ corresponding to the binary package $p$, a set of builders $b_1, b_2, \dotsc, b_k$, a cryptographic hash function $H$, a function for generating message authentication codes $MAC$, judge the reproducibility of $p$.
\end{quote}

A package judgment is initiated by a builder selecting a default value $d$ and supplying it together with a \texttt{.buildinfo} file name $n_p$ to a Hyperledger Fabric network. The judgment process is then a two-phase commit with an initial round of hidden votes followed by a round of vote reveals. By separating the vote into two phases, no two participating builders votes can influence each other even though the final votes are not anonymized. Any builder on the network may then vote on the reproducibility of the package by (I) selecting a key $k$, (II) calculating the checksum of the reproduced package $c = H(p)$, (III) encrypting the checksum $e = MAC(k, c)$ and (IV) adding the encrypted result to the Hyperledger Fabric network. If a package is not reproducible, the builder instead uses the default value $d$ instead of the checksum $c$ in the third step. At a suitable moment, the initiator can then close the first phase of the judgment process and open it up for vote reveals. A builder reveals their vote by providing the $k$ and $c$ values to the Fabric network, which then verifies the consistency of the vote. The judgment process can then be ended by the initiator. At this point, the votes are counted and the final verdict as determined by a majority of the votes is stored on the network.

\subsection{Multi-package algorithm and reward scheme}

To lift the algorithm described in section \ref{subsec:AlgorithmSingle} from working on a single package to multiple packages, a reward scheme is added to incentive builder collaboration and the number of simultaneous active judgments a builder have initialized is limited to one. The reward scheme relies on every builder having a wallet of build tokens (BT) that is stored on the network. While initializing a package judgment, the initializing builder provides a target vote level corresponding to the number of build tokens they are willing to pay for the judgment. \textbf{(minimum cost???)} A package with a trust level of 1, requires one additional builder to judge the package for the initiator to be satisfied with the judgment. The initiator then pays 1 BT to the other builder when ending the package judgment. For each added trust level, one more builder is required to judge the package. The cost per added builder goes up by one, such that with $n$ builders, the $k^{th}$ one to judge a package receives $n - k + 1$ BT from the initiator. The minimum cost for a judgment at trust level $l$ is then $1 + 2 + \dotsc + l = \frac{l(l+1)}{2}$. If a decision can be made on the reproducibility of a package, the initiator pays out the respective number of build tokens to each builder of the winning side up to the target vote level number of builders. Every builder on the winning side also receive one (1) additional build token from the system. Builders not participating in the package judgment loose one (1) build token, while those builders on the losing side of a package judgment have no changes made to their wallets. The initiator also neither receives nor looses any build tokens except for the payout of the package judgment itself. In summary, the scheme rewards participating builders and penalizes non-participating ones.
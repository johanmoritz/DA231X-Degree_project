\chapter{Introduction}
\label{ch:introduction}
% \todo[inline, backgroundcolor=kth-lightblue]{svensk: Introduktion}


% \todo[inline, backgroundcolor=kth-lightblue]{Ofta kommer problemet och problemägaren
% 	från industrin där man önskar en specifik lösning på ett specifikt
% 	problem. Detta är ofta "för smalt" definierat och ger ofta en "för smal"
% 	lösning för att resultatet skall vara intressant ur ett mer allmänt
% 	ingenjörsperspektiv och med "nya" erfarenheter som resultat. Fundera
% 	tillsammans med projektets intressenter (student, problemägare och akademi)
% 	hur man skulle kunna använda det aktuella problemet/förslaget för att
% 	undersöka någon ingenjörsaspekt och vars resultat kan ge ny eller
% 	kompletterande erfarenhet till ingenjörssamfundet och vetenskapen.\\

% 	Examensarbetet handlar då om att ta fram denna nya "erfarenhet" och på köpet
% 	löser man en del eller hela delen av det ursprungliga problemet.\\

% 	Erfarenheten kommer ur en frågeställning som man i examensarbetet försöker
% 	besvara med tidigare och andras erfarenhet, egna eller modifierade metoder som
% 	ger ett resultat vilket kan användas för att diskutera ett svar på
% 	undersökningsfrågan.\\

% 	Detta stycke skall alltså, förutom det ursprungliga "smala" problemet,
% 	innehålla  vad som skall undersökas för att skapa ny ingenjörserfarenhet
% 	och/eller vetenskap.
% }

% \todo[inline, backgroundcolor=kth-lightgreen]{The first paragraph after a heading is not indented, all of the
% 	subsequent paragraphs have their first line indented.}

This chapter describes the specific problem that this thesis addresses, the context of the problem, the
goals of this thesis project, and outlines the structure of the thesis.\\

% \todo[inline]{Give a general introduction to the area. (Remember to use appropriate references in this and all other sections.)}

% % One can use either biblatex or bibtex - set as the option for the document at the top of this file
% \ifbiblatex
% 	\todo[inline, backgroundcolor=kth-lightgreen]{We use the \emph{biblatex} package to handle our references.  We
% 		use the command \texttt{parencite} to get a reference in parenthesis, like
% 		this \textbackslash parencite\{heisenberg2015\} resulting in \parencite{heisenberg2015}.  It is also possible to include the author as part of the sentence using \texttt{textcite}, like talking about the work of \textbackslash textcite\{einstein2016\} resulting in \textcite{einstein2016}.\\
% 		This also means that you have to change the include files to include biblatex and change the way that the reference.bib file is included.}
% \else
% 	\todo[inline, backgroundcolor=kth-lightgreen]{We use the \emph{bibtex} package to handle our references.  We therefore
% 		use the command \textbackslash cite\{farshin\_make\_2019\}. For example, Farshin, \etal described how to improve LLC
% 		cache performance in \cite{farshin_make_2019} in the context of links running
% 		at \SI{200}{Gbps}.}
% \fi

% \todo[inline, backgroundcolor=kth-lightgreen]{Use the glossaries package to help yourself and your readers.
% 	Add the acronyms and abbreviations to lib/acronyms.tex. Some examples are shown below:}
% In this thesis we will examine the use of \glspl{LAN}. In this thesis we will
% assume that \glspl{LAN} include \glspl{WLAN}, such as \gls{WiFi}.


\section{Background}
\label{sec:background}
% \todo[inline, backgroundcolor=kth-lightblue]{svensk: Bakgrund}

% \todo[inline]{Present the background for the area. Set the context for your project – so that your reader can understand both your project and this thesis. (Give detailed background information in Chapter 2 - together with related work.)
% 	Sometimes it is useful to insert a system diagram here so that the reader
% 	knows what are the different elements and their relationship to each
% 	other. This also introduces the names/terms/… that you are going to use
% 	throughout your thesis (be consistent). This figure will also help you later
% 	delimit what you are going to do and what others have done or will do.}

Discussions on how to verify the lack of malicious code in binaries go at least as far back as to Ken Thompson's Turing award lecture \cite{thompson_reflections_1984} where he discusses the issues of trusting code created by others. In recent years, several attacks on popular packages within the \glsentryfull{FOSS} have been executed \cite{lamb_reproducible_2021} where trusted repositories have injected malicious code in their released binaries. These attacks question how much trust in such dependencies is appropriate. In an attempt to raise the level of trust and security in \glsentryfull{FOSS}, the \textit{reproducible builds} project \cite{reproducible_builds_project} was started within the Debian community. Its goal was to mitigate the risk that a package is tampered with by ensuring that its builds are deterministic and therefore should be bit-by-bit identical over multiple rebuilds. Any user of a reproducible package can verify that it has indeed been built from its source code and was not manipulated after the fact simply by rebuilding it from the package's \texttt{.buildinfo} file. These metadata files for reproducible builds include hashes of the produced build artifacts and a description of the build environment to enable user-side verification. The crucial link to ensure reproducibility is by this notion \texttt{.buildinfo}, which also means that a great deal of trust is assumed when using them. Current measures for validating \texttt{.buildinfo} files and their corresponding packages involve package repository managers and volunteers running rebuilderd \cite{rebuilderd_public_nodate} instances that test the reproducibility of every \texttt{.buildinfo} file added to the relevant package archive. This setup allows users to audit the separate build logs, thus confirming the validity of a particular package. However, because this would be a manual process and the different instances do not coordinate their work, it relies on the user judging on a case-by-case basis whether to trust a package or not.

Validating the aggregated results from many package builds could potentially be done through \glspl{DLT}. \glsentryshortpl{DLT} were popularized by Bitcoin \cite{di_pierro_what_2017, nakamoto_bitcoin_nodate} for crypto-currencies but has wide ranging applications in trust related domains. A distributed ledger is a log of transaction held by many different nodes on a network. Transactions are validated, ordered and added to the network by a consensus algorithm to ensure that no single or small group of nodes can act maliciously. The log itself is commonly a tree or graph of hashes which allows proving that a particular transaction has happened in an efficient manner. Because of their distributed nature, \glsentryshortpl{DLT} are however hard to test and verify. One way to go about this without loosing accuracy \textbf{Precision?} is by modeling the system with formal specification tools that can validate the properties of the system design. Even though such a model is not a true representation of the system itself,

This project seeks to reduce some of the above mentioned burden from the user while increasing their trust in the software they use by investigating possible decentralized solutions for distributing and proving the correctness of \texttt{.buildinfo} files.

\section{Problem}
\label{sec:problem}
% \todo[inline, backgroundcolor=kth-lightblue]{svensk: Problemdefinition elle Frågeställning\\
% 	Lyft fram det ursprungliga problemet om det finns något och definiera därefter
% 	den ingenjörsmässiga erfarenheten eller/och vetenskapen som kan komma ur
% 	projektet. }

% Longer problem statement\\
% If possible, end this section with a question as a problem statement.

Equivalences between human-readable source code and binaries are hard to prove \textbf{(cite)}. Likewise, the same holds when comparing the source code and hash of a binary program. A more easily proved variation of this problem is whether multiple binaries have been built from the same, potentially unknown, source. If the binaries are identical and the builder is trusted not to forge their results, the binaries can confidently be assumed to have been built the same way. The proof, though, is only as strong as the trust in the builder; who could potentially be compromised. With multiple builders, the risk is reduced and the trust will likewise increase.

Distributing the workload creates the need for a system where the build results can be aggregated. Because users have different needs, they should be able to choose their own trust models and use the packages they trust based on the build results from the different builders. With this as background, there is the question of how such a system can be designed and implemented in order to maximize user trust in that the packages they use have been derived from the authentic source code.

% Research Question
\subsection{Original problem and definition}
\label{sec:researchQuestion}
% \todo[inline, backgroundcolor=kth-lightblue]{Ursprungligt problem och definition}

The \textit{Debian Reproducible Builds} project uses \texttt{.buildinfo} files to store checksums of derived artifacts. These files can serve as proof that a package has been built from source by a particular builder. Storing the aggregated \texttt{.buildinfo} files from multiple builders in a system could increase user trust in \texttt{.buildinfo} files, and therefore in packages. This follows from the fact that even though any particular builder could act maliciously, the probability that all builders are doing so lowers for each added builder. With this in mind we ask the following question:

\begin{itemize}
	\item To what extent can distributed and decentralized storage secure the integrity of \texttt{.buildinfo} files?
\end{itemize}

\subsection{Scientific and engineering issues}
\label{subsec:scientific-issues}
% \todo[inline, backgroundcolor=kth-lightblue]{Vetenskaplig och ingenjörsmässig frågeställning}

In the process to answer the research question, the following more technical questions will have to be answered as well:

\begin{itemize}
	\item Which distributed data storage solutions are applicable for \texttt{.buildinfo} files?
	\item How can we model a relevant system for efficient evaluation of integrity preservation?
\end{itemize}

\section{Purpose}
% \todo[inline, backgroundcolor=kth-lightblue]{Syfte}
% \todo[inline]{State the purpose  of your thesis and the purpose of your degree project.\\
% 	Describe who benefits and how they benefit if you achieve your goals. Include anticipated ethical, sustainability, social issues, etc. related to your project. (Return to these in your reflections in Section~\ref{sec:reflections}.)}

% \todo[inline, backgroundcolor=kth-lightblue]{Skilj på syfte och mål! Syfte är att förändra något till det bättre. I examensarbetet finns ofta två aspekter på detta. Dels vill problemägaren (företaget) få sitt problem löst till det bättre men akademin och ingenjörssamfundet vill också få nya erfarenheter och vetskap. Beskriv ett syfte som tillfredställer båda dessa aspekter.\\
% 	Det finns även ett syfte till som kan vara värt att beakta och det är att du som student skall ta examen och att du måste bevisa, i ditt examensarbete, att du uppfyller examensmålen. Dessa mål sammanfaller med kursmålen för examensarbetskursen.
% 	}

As society relies more and more heavily on software and digital infrastructure, threats to those technologies are increasingly more important to mitigate. By supplying additional safeguards to the way we manage software, we can make it harder for malicious actors to take advantage of users.

One current way of managing software in a safe manner is to first download its source code and then building it on our own machines. This way, we can be confident in that we are running the software we intend to run. Such a method, however, is time consuming for the user. The purpose of this project is to give alternative solutions with a focus on user trust while not relying on users' building packages themselves. Furthermore, this project seek to increase trust and security in \glsentrylong{FOSS} and reduce the risk of supply-chain attacks on package archives.


\section{Goals}

The main goal of this project is to formulate a plan for how to store \texttt{.buildinfo} files in such a way that their integrity is maintained. This involves understanding the purpose \texttt{.buildinfo} files and the context they exist within. The storage plan should be formulated based on this context and written as a formal specification.

% \todo[inline, backgroundcolor=kth-lightblue]{Mål}
% \todo[inline, backgroundcolor=kth-lightblue]{Skilj på syfte och mål. Syftet är att åstakomma en förändring i något. Målen är vad som konkret skall göras för att om möjligt uppnå den önskade förändringen (syfte). }

% \todo[inline]{State the goal/goals of this degree project.}

% The goal of this project is XXX. This has been divided into the following three sub-goals:
% \begin{enumerate}
% 	\item Subgoal 1 \todo[inline, backgroundcolor=kth-lightblue]{för att tillfredsställa problemägaren – industrin?}
% 	\item Subgoal 2\todo[inline, backgroundcolor=kth-lightblue]{för att tillfredsställa ingenjörssamfundet och vetenskapen – akademin) }
% 	\item Subgoal 3\todo[inline, backgroundcolor=kth-lightblue]{eventuellt, för att uppfylla kursmålen – du som student}
% \end{enumerate}

% \todo[inline]{In addition to presenting the goal(s), you might also state what the deliverables and results of the project are.}

\section{Research Methodology}
% \todo[inline, backgroundcolor=kth-lightblue]{Undersökningsmetod}
% \todo[inline, backgroundcolor=kth-lightblue]{Här anger du vilken vilken övergripande undersökningsstrategi eller metod du skall använda för att försöka besvara den akademiska frågeställning och samtidigt lösa det e v ursprungliga problemet. Ofta kan man använda "lösandet av ursprungsproblemet" som en fallstudie kring en akademisk frågeställning. Du undersöker någon intressant fråga i "skarpt" läge och samlar resultat och erfarenhet ur detta.\\
% 	Tänk på att företaget ibland måste stå tillbaka i sin önskan och förväntan på projektets resultat till förmån för ny eller kompletterande ingenjörserfarenhet och vetenskap (ditt examensarbete). Det är du som student som bestämmer och löser fördelningen mellan dessa två intressen men se till att alla är informerade. }
% \todo[inline]{Introduce your choice of methodology/methodologies and method/methods – and the reason why you chose them. Contrast them with and explain why you did not choose other methodologies or methods. (The details of the actual methodology and method you have chosen will be given in Chapter~\ref{ch:methods}. Note that in Chapter~\ref{ch:methods}, the focus could be research strategies, data collection, data analysis, and quality assurance.)\\
% 	In this section you should present your philosophical assumption(s), research method(s), and research approach(es).}

The project is in three phases. The initial phase is a pre-study focusing on reproducible builds, \glsentrylongpl{DLT} and formal specification. Its purpose is researching possible technologies, solutions and evaluation methods for solving the issues mentioned under section \ref{subsec:scientific-issues}. With this initial phases finished, an appropriate storage strategy for ensuring the integrity of \texttt{.buildinfo} files and a methodology for modeling such a storage system is decided. The last phase involves producing and evaluating the model of said system to find an answer to the original research question stated in \ref{sec:researchQuestion}.

\section{Delimitations}
% \todo[inline, backgroundcolor=kth-lightblue]{Avgränsningar}
% \todo[inline]{Describe the boundary/limits of your thesis project and what you are explicitly not going to do. This will help you bound your efforts – as you have clearly defined what is out of the scope of this thesis project. Explain the delimitations. These are all the things that could affect the study if they were examined and included in the degree project.}

While this project utilizes \texttt{.buildinfo} files and reproducible builds as its core problem domain, no builds or \texttt{.buildinfo} files are necessarily produced during it. More specifically, the relevant part of \texttt{.buildinfo} files in terms of the project are their meta information and context in the software ecosystem. Their actual content and semantics are mainly irrelevant for the project and is represented in an abstract manner in any implementation or artifact.

% \section{Structure of the thesis}\todo[inline, backgroundcolor=kth-lightblue]{ Rapportens disposition}
% \label{sec:structure}
% Chapter~\ref{ch:background} presents relevant background information about xxx.  Chapter~\ref{ch:methods} presents the methodology and method used to solve the problem. …
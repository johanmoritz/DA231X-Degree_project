\chapter{Discussion}
% \todo[inline]{This can be a separate chapter or a section
% 	in the previous chapter.}
\label{ch:discussion}
% \todo[inline, backgroundcolor=kth-lightblue]{Diskussion\\
% 	Förbättringsförslag?}

\section{Model bounds}

While the model checking runs successfully in the scenarios laid out in section \ref{sec:model_evaluation}, these are all artificially limited in their size. As can be noted from the results, the time it takes to model check each scenario grows significantly with the bounds of the model. The upper limit of these scenarios are therefore in practice bounded by compute time. The consequence of this limit is that the model is not verified to work successfully with higher goals or in larger networks. Although there is little to suggest that such an extension would significantly change the models behavior in practice, the results could very well be improved or strengthened by model checking on larger and more complex scenarios.

\section{Model refinement}

Another way to improve the results would be to refine the model itself. Currently, its focus is on the interactions between system actors and the reward system while it leaves the details for how to cryptographically handle the voting process completely. It also disregards the Fabric transactional process in favour of simpler model. By modelling the Fabric API and transactions, the refinement would be closer to an implementation. However, it would most likely also increase model checking duration. At that point, an argument could be made that testing an implementation could be more productive rather than making the model more complex.


\section{User behavior}

The assumptions on user behavior in this report have overall been minor and as limited as possible. However, there could be important differences between how the system actors are modelled and how actual users of the system would behave. While it is outside of the scope of this work, it would be interesting to study how well the incentive mechanisms and behavior used represent the relevant users. This is especially true for modelling malicious system actors, which the result mostly disregard. While a Hyperledger Fabric network has strong security properties and allows a certain fraction of malicious nodes, the result does not particularly address whether a system implementation as analyzed would open up other security vulnerabilities due to unforeseen user patterns.

\section{Fabric security properties}

\section{Improved voting}

In the single-package algorithm described in section \ref{subsec:AlgorithmSingle}, the values to represent positive and negative votes respectively were all the same and selected at the start of a package judgment. This means that while the judgment decision can be traced on the Fabric blockchain, there would be no proof that all builders have built the exact same package. Depending on the situation, such a relaxed view could be preferable though, as it would simplify builder setup coordination and the reproducibility is tested in different environments. If the wish would be to strengthen the traceability, the builders could be asked to supply a proof for a positive vote by adding the checksum of the built package to their vote. With such a change, all positive votes should be equal after they have been revealed without needing any centralized coordination on the allowed votes.

\section{Key selection}

While a system user's number of choices has been minimized to simplify analysis, the matter of key selection for the \gls{HMAC} usage was up to the user to decide on their own. This could be a possible entry point for users to collude by using pre-determined or simple keys. If only considering non-malicious users, this can possibly be resolved by generating keys inside chaincode, thus ensuring that all keys are created in the same standard way. 

\todoinline{The builders should only use a hash for YES or NO. This more general voting scheme (with multiple different votes) should be added as a discussion point.}

\todoinline{Currently, the security evaluation of the single-judgment algorithm is very limited. We make no argument as to how long the votes and keys used for hmac should be.}

\todoinline{Builders can choose what ever terrible key they want when using HMAC. This is fine because we assume they cannot communicate outside the network. In practice though, it is a problem. }
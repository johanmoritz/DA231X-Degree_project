\chapter{Conclusions and Future work}
% \label{ch:conclusionsAndFutureWork}
% \todo[inline, backgroundcolor=kth-lightblue]{Slutsats och framtida arbete}

% \todo[inline]{Add text to introduce the subsections of this chapter.}

\section{Conclusions}
\label{sec:conclusions}
% \todo[inline]{Describe the conclusions (reflect on the whole introduction given in Chapter 1).}
% \todo[inline, backgroundcolor=kth-lightblue]{Slutsatser}


% \todo[inline]{Discuss the positive effects and the drawbacks.\\
% 	Describe the evaluation of the results of the degree project.\\
% 	Did you meet your goals?\\
% 	What insights have you gained?\\
% 	What suggestions can you give to others working in this area?\\
% 	If you had it to do again, what would you have done differently?}

% \todo[inline, backgroundcolor=kth-lightblue]{Träffade du dina mål?\\
% 	Vilka insikter har du fått?\\
% 	Vilka förslag kan du ge till andra som arbetar inom detta område?
% 	Om du hade att göra igen, vad skulle du ha gjort annorlunda?}

\section{Limitations}
\label{sec:limitations}
% \todo[inline]{What did you find that limited your
% 	efforts? What are the limitations of your results?}
% \todo[inline, backgroundcolor=kth-lightblue]{Begränsande faktorer\\
% 	Vad gjorde du som begränsade dina ansträngningar? Vilka är begränsningarna i dina resultat?}

\section{Future work}
\label{sec:futureWork}
% \todo[inline]{Describe valid future work that you or someone else could or should do.\\
% 	Consider: What you have left undone? What are the next obvious things to be done? What hints can you give to the next person who is going to follow up on your work?
% }
% \todo[inline, backgroundcolor=kth-lightblue]{Vad du har kvar ogjort?\\
% 	Vad är nästa självklara saker som ska göras?\\
% 	Vad tips kan du ge till nästa person som kommer att följa upp på ditt arbete?
% }


% Due to the breadth of the problem, only some of the initial goals have been
% met. In these section we will focus on some of the remaining issues that
% should be addressed in future work. ...

% \subsection{What has been left undone?}
% \label{what-has-been-left-undone}

% The prototype does not address the third requirment, i.e., a yearly
% unavailability of less than 3 minutes, this remains an open problem. ...

% \subsubsection{Cost analysis}

% The current prototype works, but the performance from a cost perspective makes
% this an impractical solution. Future work must reduce the cost of this
% solution, to do so a cost analysis needs to first be done. ...

% \subsubsection{Security}

% A future research effort is needed to address the security holes that results
% from using a self-signed certificate. Page filling text mass. Page filling
% text mass. ...


% \subsection{Next obvious things to be done}

% In particular, the author of this thesis wishes to point out xxxxxx remains as
% a problem to be solved. Solving this problem is the next thing that should be
% done. ...

\section{Reflections}
\label{sec:reflections}
% \todo[inline]{What are the relevant economic, social,
% 	environmental, and ethical aspects of your work?
% }
% \todo[inline, backgroundcolor=kth-lightblue]{Reflektioner}
% \todo[inline, backgroundcolor=kth-lightblue]{Vilka är de relevanta ekonomiska, sociala, miljömässiga och etiska aspekter av ditt arbete?}


% One of the most important results is the reduction in the amount of
% energy required to process each packet while at the same time reducing the
% time required to process each packet.

% % The thesis contributes to the \gls{UN}\enspace\glspl{SDG} numbers 1 and 9 by
% xxxx.